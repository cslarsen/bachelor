\chapter{Introduction}

\section{Software defined networking}

{\em Software defined networking} emerged from research at Berkeley and
Stanford around 2008 as a way to enable networks to be defined and managed
using software.

Central to the idea is to split the {\em control} and {\em data} planes.
While routers traditionally contained both, one would now implement the
controller in {\em software} running on servers and have them talk to each
other using a predetermined protocol.

One such protocol is {\em OpenFlow}.  It allows the two planes to interact
using a predetermined set of commands.  The controller configures so--called
{\em flow tables} in the data plane which contain matching patterns and
actions for incoming packets.  These tables are designed for high--speed
routing, and are thus quite simple.  On the other hand, the controller can
be arbitrarily complex.

To maintain high networking speeds, one wants to restrict the interaction
between the data and control planes to a minimum.  This happens when the
data plane is able to route packets entirely on its own.  The control plane
update the flow tables as seldomly as possible.

The data plane can also forward packets to the control plane for routing
decisions. The extreme case is when {\em all} packets are forwarded to the
control plane.  While this allows very advanced behaviour --- such as deep
packet inspection, for instance --- it obviously comes at the cost of
lower networking performance.  Therefore, one must be diligent in designing
such a network.

Today, several SDNs have been deployed in production networks and power some
of the biggest services on the internet\footnote{Examples are Google's {\em
Chubby} distributed lock service and the {\em Megastore} distributed,
high--availability storage system.}.

\section{Paxos}

{\em Paxos} is a family of distributed, fault--tolerant consensus
algorithms.  It allows network nodes to reach {\em agreement} even in the
face of intermittent network failures.  For example, one can design a
database system using Paxos to make sure that transactions are executed in
the same order on all nodes.

Originally published in 1989 by Leslie
Lamport\cite{Lamport:1998:PP:279227.279229}, Paxos has spawned numerous
extensions like Byzantine tolerance and so on.

\section{A Paxos--enabled software defined network}

Our aim is to build an efficient, {\em Paxos--enabled software defined
network} where nodes are able to leverage the guarantees of Paxos without
needing to handle the details.

For simplicity, we will constrain our scope to Paxos {\em phase two} where
we have steady--state flow with no failures.

We will implement this in progressive stages:

\begin{itemize}
\item Stage 1: Implement Paxos entirely on the controller.
\item Stage 2: Move parts of Paxos down from the controllers to the
switches.
\item Stage 3: Look for possible expansion of the OpenFlow protocol to
facilitate running Paxos on the switches.
\end{itemize}

Stage 1 should be the easiest to implement, and will be used as a baseline
for comparing networking performance with later stages.

Additionally, we will implement Paxos entirely on the network nodes, using
the same topology, to get an idea of how much of a performance hit we take
for running it on the controller.

In stage 2, we will look at how we can improve efficiency by moving parts of
the algorithm down into the switches using the flow tables.

Finally, in stage 3 we will investigate whether it would make sense to
enhance the OpenFlow protocol with new primitives to build fast
networking services such as Paxos.

{\em 
  Our hypothesis is that by moving parts of the Paxos implementation down to
  the switches, we will get a more performant system than by running Paxos
  either on the controller or on the network nodes.
}

The thesis will therefore be a study of {\em feasibility}.
