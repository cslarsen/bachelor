\chapter{Introduction}
\label{chapter:introduction}

With the emergence of software-defined networking
\cite{Casado:2005:VNS:1047344.1047383}, the switch becomes a much
more central component in the network infrastructure.

That is, the switch can be programmed to perform a mude wider range of
functions that previously could not be done, or had to be done on end-hosts
or in the router.
%
This includes support for packet filtering (firewall), intrusion detection
and much more.

Emerging from research at Berkeley\index{Berkeley} and
Stanford\index{Stanford} around 2008, the basic idea was to decouple the
control plane\index{control plane} from the \textit{forwarding plane} (or
\textit{data plane}).
%
The forwarding plane is where packets are transferred from one point to
another, perhaps changing some header fields along the way.
%
By design, the operations in the forwarding plane are simple to enable high
performance.
%
The decisions about \textit{how} the forwarding plane should function is
made in the control plane---by a controller---and is much more advanced in
comparison.
%
For a forwarding plane to operate efficiently, the controller must have
access to information such as the network topology, an overview of traffic
and so on.  It can then program the forwarding plane accordingly.

OpenFlow \cite{openflow-1.0} is one way to use software-defined
networking.  It specifies a communication protocol between a controller and
a switch.
Using OpenFlow, one can write controllers in practically any programming
language, making it easy to build networks in a much more dynamic fashion
than before.

Although invented quite recently, software-defined networking has already
seen heavy use both in academia and industry.
%
Google\index{Google}, for instance, are
using OpenFlow to ease deployment and increase utilization in their backbone
networks\index{backbone network} \cite{crabbe2012sdn} and Stanford has deployed several
OpenFlow-controlled networks on their university network.

Another trend in the computing industry is that more and more services are
being moved to the cloud, reaching a larger number of users.
%
Thus, it becomes even more important that these services remain available
despite failure of individual machines running those services.
%
This calls for replicating these services on several machines.
%
A huge body of research in the distributed computing community has been
devoted to coming up with protocols to guarantee strong consistency among
the replicas of such a service.
%
The most cited of these protocols is the Paxos protocol
\cite{Lam01,Lamport:1998:PP:279227.279229}.

Paxos is a
family of distributed, fault-tolerant consensus algorithms.  It allows
network nodes to reach \textit{agreement} even in the face of intermittent
network failures.  For example, one can design a database system using Paxos
to make sure that transactions are executed in the same order on all nodes.

Most general-purpose implementations of Paxos are built as middleware
software that must be integrated into server code.  This has several
drawbacks.  For instance, it operates at the upper level of the networking
stack---at the application level---and away from the central switching
points in a network.  Another drawback is that one has to tailor Paxos
specifically for each and every service.

In this thesis we want to investigate how we can build Paxos at a networking
level, closer to its central parts and in particular, if this can be done
transparently, making it possible to offer Paxos to services not originally
designed for distributed operation.
%
Furthermore, to demonstrate its applicability, we want to build a system of
replicated services, where each \textit{end-host} (running general purpose
software services such as log-servers, etc.) is replicated by receiving the
same packets from clients.
%
Using Paxos, we ensure that the services receive the packets in-order, even
in the face of failures.

There may be several benefits in doing this, and we will return to them
later in this thesis, but list the three most important here:

\begin{itemize}
  \item \textbf{Transparency:} We guarantee consistent in-order delivery of
  network packets at all replicas.  This has the benefit that network
  applications can be replicated without the need for complicated Paxos
  logic. While this is usually handled by a middleware framework, we can
  avoid imposing a specific API on the application developer.  For example,
  frameworks typically impose a language specific API, preventing users of
  other languages from leveraging the replication service.
  %
  We ignore the need for state transfer in this thesis.

  \item \textbf{Performance:} By implementing Paxos in the switch, we
  expect to see a performance improvement. This is because, assuming the
  latency between the switches is low, we can avoid the extra messages and
  latency of traversing the links and IP stack of the end-hosts running the
  replicas of the services.

  \item \textbf{Traffic reduction:}
  Using a setup with ``Paxified'' switches, every application that wishes to
  have their messages totally ordered can use that functionality offered by
  the switches.
  %
  Thus, we can leverage one setup to provide several replicated
  services with ordered message delivery. Therefore, instead of having each
  replicated service runing its own Paxos protocol, they can all receive
  ordered messages from the switches. While this still requires Paxos
  messages exchange between switches, it can reduce the traffic on the
  links between the switches and replicas.

\end{itemize}

There are a few drawbacks and restrictions that are worth mentioning.

The performance improvement we expect to see is fairly small compared to
wide-area network latencies.  Thus, it may not make sense to place Paxos
logic in the switches that communicate over wide-area links. That said,
there are several works in which one runs two separate Paxos
protocols, one within each data center, and another between data
centers.
In such a configuration, it would definitely make sense to use in-switch
Paxos deployments within the data center, even though the other Paxos does
not.

While most of the time a single switch operates as an autonomous unit, to
implement Paxos in the switch, we need to impose some restrictions on the
topology between the machines.

A drawback with implementing Paxos in the siwtch is that a physical switch
typically has a very restricted instruction set available, and thus
implementing Paxos at this low-level can be very challenging and prone to
unforeseen issues.

Despite these restrictions, we believe that the benefits can be of value to
a wide range of deployments where it is desirable to provision enough
replication to ensure high availability of services.

To reduce the scope of this thesis,
we will constrain our scope to a few primitives of
classic crash Paxos\index{Paxos!classic crash} in \textit{phase two},
where we have steady-state
flow\index{Paxos!steady-state flow} with no failures\index{Paxos!failure}.

Most importantly, ths is a study of \textit{feasibility}.  We want to
explore whether implementing a distributed protocol at lower networking
levels is doable, viable and whether it can be offered transparently to the
services.
