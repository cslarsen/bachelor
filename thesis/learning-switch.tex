\section{An L2 learning switch in OpenFlow}

When you write an OpenFlow controller, the flow table is empty and all
packets will by default be delivered to the controller.

The controller must then decide what to do with the packets.  If we don't
implement any sort of forwarding behvaiour for the packets, none of the
hosts will be able to communicate.

So our system will need a forwarding mechanism at the bottom of the Paxos
networking capabilities.  The simplest system is just to implement a
\textit{hub}:  For each packet coming in to the switch, flood it (or,
\textit{rebroadcast}) to all ports, and let each connected host decide 
to receive packets meant for them (algorithm \ref{algorithm:l2.hub}).

\begin{algorithm}
  \begin{algorithmic}
    \On{Ethernet packet $e$}{port $p$}
      \State \textbf{flood} $p$ \Comment{Send packet out on \textit{all} ports}
    \EndOn
  \end{algorithmic}
  \caption{An L2 hub algorithm}
  \label{algorithm:l2.hub}
\end{algorithm}

A slightly better approach is to implement an \ac{L2} learning switch.
The difference from the flood--to--all hub above is that we create a table
that maps MAC--addresses to ports and then forward each packet to a single
port.  We then achieve less traffic on the network.

As we build up this table we could also install flow table entries so that the
switch will be able to forward packets by itself.  This is indicated in
algorithm \vref{algorithm:l2.learning.switch} with
\textbf{add.flowtable.entry}, but is an optional step.

\begin{algorithm}
  \begin{algorithmic}
    \State $M \gets \emptyset$\Comment{Set containing $\langle
      \text{mac--address},\ port \rangle$--tuples}
    \State
    \On{Ethernet packet $e$}{port $p$}
      \State $m_{dst} \gets \text{destination MAC address in}\ e$
      \State $m_{src} \gets \text{source MAC address in}\ e$
      \State
      \State $M \gets M \cup \langle m_{src},\ p \rangle$ \Comment{Learn
        which port $m_{src}$ can be reached on}
      \State
      \State \textbf{add.flowtable.entry}(for ethernet packets to
        $m_{src}$, \textbf{forward} to port $p$)
      \State
      \If{$\langle m_{dst},\ \cdot\ \rangle \in M$} \Comment{Do we
          \textit{also} know the destination port?}
        \State \textbf{forward} $p$ \textbf{to} destination port for $m_{dst}$ in $M$
      \Else
        \State \textbf{flood} $p$ \Comment{Act as hub; send packet to all ports}
      \EndIf
    \EndOn
  \end{algorithmic}
  \caption{Algorithm for an L2 learning switch.}
  \label{algorithm:l2.learning.switch}
\end{algorithm}

As you can see, algorithm \ref{algorithm:l2.learning.switch} will need to
run at least twice before it will know both the source and destination ports
for two MAC--addresses.  If we send an \textit{\acs{ICMP} ping packet} from
host $a$ to $b$, the switch running the algorithm will first learn which port
$a$ is on, and then flood the packet out all ports (it doesn't know which
port $b$ is on, yet).

$b$ will then receive the packet\footnote{The other hosts' networking stack
will simply drop the packet, as it's not for them---unless their \acs{NIC} is
running in \textit{promiscuous mode}, capturing all packets.} and send an \acs{ICMP}
ping reply packet.  When this reaches the switch, it will learn which port
$b$ is on and is now able to do a packet forwarding instead of a flood,
because it knows which port $a$ is on (the packet from $b$ has $b_{address}$ as
source MAC--address and $a_{address}$ as destination MAC--address).

We can also install rules in the OpenFlow flow table so that
subsequent packets to these two hosts will be forwarded automatically to
their respective port---without any interaction from the controller.  It
also means that the controller will not see those packets anymore.  As
mentioned elsewhere, each flow table entry has an associated set of idle and
hard timeout counters.  We've not indicated values for these here, but
typically one sets the idle timeout to 10 seconds and the hard timeout to 60
seconds.  This means that we have to keep adding the rules again and again,
but from now that will be done automatically by the algorithm\footnote{The
timeouts help keep the flow table from going full.}

Finally, one must realize that it doesn't matter if the ports are connected
\textit{directly} to hosts with the associated MAC--addresses.  Even if the
ports are links to other networks, we know that a MAC--address has been seen
coming from this port, and should therefore be reachable, somehow, on that
port.

This algorithm has been implemented in Python using the POX controller, with
the exception that we don't install forwarding flow table
entries\footnote{This is just for implementation simplicity, since we also
need to install flows that react on PAXOS messages.  In a production
system, we would install L2 forwarding with a lower priority than those
entries reacting on PAXOS--messages, otherwise the PAXOS handling code would
never be run.}.
