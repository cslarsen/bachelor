\section{An L2 Learning Switch in OpenFlow}
\label{chapter:l2.learning.switch}

When you write an OpenFlow controller, the flow table is empty and all
packets will by default be delivered to the controller via an
\textit{upcall}\index{upcall}.

The controller must then decide what to do with the packets.  If we don't
implement any sort of forwarding behaviour for the packets, none of the
hosts will be able to communicate.

So our system will need a forwarding mechanism below the level where Paxos
operates.
%
The simplest system is just to implement a \textit{hub}\index{hub}:  For
each packet coming in to the switch, flood it (or,
\textit{rebroadcast}\index{rebroadcasting}) to all ports except the
input port.
%
Each node receiving packets will silently drop those who are not addressed
to them (algorithm \ref{algorithm:l2.hub}).

\begin{algorithm}
  \begin{algorithmic}
    \On{Ethernet frame $e$}{port $p$}
      \State \textbf{flood} $e$ \Comment{Send packet out on all ports except
        $p$}
    \EndOn
  \end{algorithmic}
  \caption{An L2 hub algorithm}
  \label{algorithm:l2.hub}
\end{algorithm}

A slightly better approach is to implement an \ac{L2} learning
switch\index{switch!L2 learning}\index{learning switch}.
The difference from the flood-to-all hub above is that we create a table
that maps MAC-addresses to ports and then forward each packet to a single
port.  We then achieve less traffic on the network.

As we build up this table we could also install flow table entries so that the
switch will be able to forward packets by itself.  This is indicated in
algorithm \vref{algorithm:l2.learning.switch} with
\textbf{add.flowtable.entry}, but is an optional step.

\todo{fiks algoritmen, inneholder en liten feil}
\begin{algorithm}
  \begin{algorithmic}
    \State $M \gets \emptyset$\Comment{Map of $address \rightarrow port$}
    \State
    \On{Ethernet frame $e$}{port $p$}
      \State $M \gets M \cup \langle e_{src},\ p \rangle$ \Comment{Learn
        port $p$ for $e_{src}$ (source MAC-address)}
      \State
      \State \textbf{add OpenFlow rule }(for Ethernet frames from $e_{dst}$ to
        $e_{src}$, forward to port $p$)
      \State
      \If{$\{ \exists q : \langle e_{dst},\ q \rangle \in M \}$}
        \Comment{See if we know the destination port $q$}
        \State \textbf{forward} $e$ \textbf{to} port $q$ for $e_{dst}$ in $M$
      \Else
        \State \textbf{flood} $e$ \Comment{Act as hub; rebroadcast packet
          $e$ to all ports}
      \EndIf
    \EndOn
  \end{algorithmic}
  \caption{An L2 learning switch algorithm for an OpenFlow controller}
  \label{algorithm:l2.learning.switch}
\end{algorithm}
\todo{Fiks membership test sjekk her, skal en feks bruke $\exists$ eller noe
  sånt, og er wildcard-operator bruk korrekt?}

As you can see, algorithm \ref{algorithm:l2.learning.switch} will need to
run at least twice before it will know both the source and destination ports
for two MAC-addresses.

Assume that we are running algorithm \ref{algorithm:l2.learning.switch} on
an OpenFlow controller connected to a switch whose flow table is empty.
%
Recall that packets who do not match any flow table entries are upcalled to
the controller.

If we now send an \textit{\acs{ICMP} ping packet}\index{ping} from host $a$ to
host $b$, the controller will learn which port host $a$ is on, but will not know
the port for host $b$ yet.
%
It will then flood the packet out on all ports except the input port.\footnote{
Hosts who receive unsolicited packets will silently drop them,
unless they are running in \textit{promiscuous mode} or similar, capturing
all packets.}

Host $b$ will then receive the packet and send an \acs{ICMP} ping reply packet
addressed to host $a$.
%
When the controller receives the reply, it knows the port for host $a$ and
can then \textit{forward} the packet to its port instead of flooding it.
%
Simultaneously, it will learn which port host $b$ is on.
%
At this point the controller can decide to install forwarding flows in the
switch so that packets are automatically forwarded.
%
One such flow can be to match on source address $a$, destination address
$b$ with the action to forward to the port for host $b$.
%
Another flow can handle the reverse case.

Algorithm \ref{algorithm:l2.learning.switch} uses a well-known
implementation technique for learning switches.
%
The one we have implemented is based on the one given in the OpenFlow
tutorial\index{OpenFlow!tutorial} \cite{github:pox.tutorial}.
%
There are additional checks that we do not perform, like not installing
flows that echo packets to the incoming port (see, e.g., the pseudo-code in
figure 3 of \cite{Canini:2012:NWT:2228298.2228312}).

There is another very important point to be made here:
%
\textit{Packets that are handled by a flow may not be seen by the
  controller.}
%
In the above algorithm, it could be tempting to install a forwarding flow as
soon as we know which port a host is on.
%
Now consider the ping example involving hosts $a$ and $b$.
%
The controller would learn the port for host $a$ and install a forwarding
flow.
%
But when host $b$ replies, the switch would silently forward the packet to
host $a$.
%
The controller would never learn which port $b$ is on, until host $b$ sends
a packet addressed to someone other than host $a$.
%
The lesson is that building controller algorithms may have corner cases that
are not easily recognized.

