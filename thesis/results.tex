\chapter{Results}

\section{TCP Replication}
\label{chapter:tcp.replication}

\todo{Skriv om våre forsøk og resultater for tcp-replikasjon.
Det er kan gjøres med en \textit{faux} ende-til-ende forbindelse
mellom klient og CONTROLLER, men det krever full emulering av
TCP-protokollen, konsolidering med hostene, unngå race conditions,
og dessuten må en sørge for å mappe ting som port-numre for hver
  host mot en utadvendt, enhetlig port, samme med visse andre
  verdier i tcp.  Det ER NOK MULIG, men vi har ikke sett mer
   på det.  Det er sannsynligvis en studie i seg selv.
   Kan også nevne artikler om TCP Multicasting. Eller kanskje
   se på om MPTCP kunne funke.}

\todo{Teksten under, skriv om}

Skal poengtere at Paxos på dette greiene ikke kan funke å ALLE typer
tjenester. DEt kommer helt an på hvordan app-level protokollene er designed.
VI ser bare pakker, og kan bare operere p ådet nivået. Vi kan IKKE se inni
pakkene og kode inn forståelse for hver protokoll's design, feks hvis vi
skal bruke mysql så ønsker vi ikke å måtte kode inn spesialhåndtering for
denne. Men sett at den plutselig ber én server om å kontakte en annen, eller
tenk på transaksjoner eller lignende kompliserte tilfeller, da er det ikke
sikkert at vår approach vil funke.  Denne delen bør forøvrig flyttes til
DISCUSSION. (sammen med teksten under, som må skrives om)

For our purposes, we will assume that the services
on these hosts are \textit{deterministic}\index{deterministic}
in the sense that the input uniquely determine the
state of the service after being processed---if two hosts running the
same service receive the exact same packet, their state will be
identical after having processed it.  This is a prerequisite for our
system.  The OpenFlow switches, running Paxos, will only make sure that
packets are delivered in the \textit{same order} to the end-hosts.
