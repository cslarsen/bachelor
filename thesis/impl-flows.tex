\section{Extending the OpenFlow Specification}

As discussed earlier in chapter \ref{chapter:openflow.design}, it would be
impractical to attempt to use existing actions in the OpenFlow specification
to implement the Paxos algorithm.
%
The OpenFlow specifications, as of version 1.4 \cite{openflow-1.4}, are
backward-compatible, meaning that a newer OpenFlow version will support all
features in older ones.  We will therefore choose to extend version 1.0
\cite{openflow-1.0}, because it was the first public version and therefore
the most widely supported.

The idea is to add a new \textit{Paxos action} with a parameters 
specifying whether to run the \textit{On Client}, \textit{On Accept} or 
\textit{On Learn} parts of the Paxos algorithm given in chapter
\vref{ch:simplifying.paxos}.
%
As shown in \vref{chapter:openflow.background}, we can then specify
precisely what kind of events we want to trigger Paxos ordering for and
combine that with other actions, such as which port the output should go to.

The part of the specification we need to extend is the 
\textit{Flow Action Structures} \cite[pp.~21--22]{openflow-1.0},
and it will be an \textit{optional} action \cite[pp.~3--6]{openflow-1.0}.
%
Listing \ref{listing:ofp10.action.type} shows the modification made to
the C\index{C} enumeration type \texttt{ofp10\_{}action\_{}type} from the
Open vSwitch source
code.\footnote{\texttt{ovs/include/openflow/openflow-1.0.h}}
%
The listing is identical to the official specification, except for the
number suffix in \texttt{OFPAT10}.

\begin{lstlisting}[
  caption={Adding the \texttt{OFPAT10\_{}PAXOS} action to the OpenFlow
           specification},
  label={listing:ofp10.action.type}]
enum ofp10_action_type {
    OFPAT10_OUTPUT,             /* Output to switch port. */
    OFPAT10_SET_VLAN_VID,       /* Set the 802.1q VLAN id. */
    OFPAT10_SET_VLAN_PCP,       /* Set the 802.1q priority. */
    OFPAT10_STRIP_VLAN,         /* Strip the 802.1q header. */
    OFPAT10_SET_DL_SRC,         /* Ethernet source address. */
    OFPAT10_SET_DL_DST,         /* Ethernet destination address. */
    OFPAT10_SET_NW_SRC,         /* IP source address. */
    OFPAT10_SET_NW_DST,         /* IP destination address. */
    OFPAT10_SET_NW_TOS,         /* IP ToS (DSCP field, 6 bits). */
    OFPAT10_SET_TP_SRC,         /* TCP/UDP source port. */
    OFPAT10_SET_TP_DST,         /* TCP/UDP destination port. */
    OFPAT10_ENQUEUE,            /* Output to queue. */
    OFPAT10_PAXOS,              /* Extension: Run Paxos algorithm. */
    OFPAT10_VENDOR = 0xffff
};
\end{lstlisting}

The Paxos action has only one parameter: Which part of algorithm
\ref{ch:simplifying.paxos} to run.  The structure of this parameter is given
in listing \ref{listing:ofp10.action.paxos} and its its possible values are
defined in table \ref{table:paxos.event.codes}.
%
All action structures are required to start with the \texttt{type} and
\texttt{len} fields.

\begin{lstlisting}[
  caption={The \texttt{OFPAT10\_{}PAXOS} parameters},
  label={listing:ofp10.action.paxos}]
struct ofp10_action_paxos {
    ovs_be16 type;            /* Required: OFPAT10_PAXOS. */
    ovs_be16 len;             /* Required: Length is 8. */
    ovs_be32 paxos_event;
};
OFP_ASSERT(sizeof(struct ofp10_action_paxos) == 8);
\end{lstlisting}

As we can see, \texttt{paxos\_{}event} is encoded as a big-endian, unsigned
32-bit integer.
%
Its possible values are given in table \ref{table:paxos.event.codes}.

\begin{table}[H]
  \centering
  \begin{tabular}{|c|l|c|l|}
    \hline
      \textbf{Value} &
      \textbf{Meaning} &
      \textbf{Algorithm} &
      \textbf{\texttt{ovs-ofctl} argument}
      \\

    \hline
      \texttt{0x7A01} &
      Run ``On Accept'' &
      \ref{algorithm:paxos.simple.acceptor} &
      \texttt{paxos:onaccept}
      \\

    \hline
      \texttt{0x7A02} &
      Run ``On Learn'' &
      \ref{algorithm:paxos.simple.learner} &
      \texttt{paxos:onlearn}
      \\

    \hline
      \texttt{0x7A40} &
      Run ``On Client'' &
      \ref{algorithm:paxos.simple.client} &
      \texttt{paxos:onclient}
      \\

    \hline
  \end{tabular}
  \caption{Possible values for \texttt{paxos\_{}event} in listing
           \ref{listing:ofp10.action.paxos}}
  \label{table:paxos.event.codes}
\end{table}

The values in table \ref{table:paxos.event.codes} have been chosen
to correspond to the Ethernet types given
in table \vref{table:paxos.ethernet.type.encoding}, although they
could have been simply zero, one and two.
%
The last column contains the command-line arguments that will be
accepted by \texttt{ovs-ofctl} when adding flows.

Because of the thesis scope, we have only added a single action parameter
\texttt{paxos\_type}.
%
In a production environment, however, one would likely need several more.
%
For example, it could be useful to distinguish between different
\textit{sets} of Paxos nodes so they could operate independently of each
other on the same network.
%
Here, we have only \textit{one} set of Paxos nodes who all have the same
leader.

\todo{Ta med:
  - the big picture, er det egentlig korrekt å gjøre en sånn endring
  i openflow? altså vi har ikke gjort dette for at vi foreslår at
  openflow faktisk --- for alle --- implementerer denne saken, men
  vi bruker det som piggy-backing for å teste ut dette på en
  komponerbar måte, så sånn sett så er det nydelig, og det kan
  lett testes med simulator, controllere osv for de støtter jo
  alleerede of
  - vedr ordering, vi har jo nå en kø som sender ut in order,
   dette er litt på kant med spekken (faktisk egentlig ikke,
       spekken sier ordering ikke er spesifisert, men vi har
       jo en side-effekt her som går kontra spekken, siden
       ting plutselig kan sendes ut senere).}

\section{Modification of Open vSwitch}

\todo{
Ta med:  hvorfor vi må endre ovs,
Features bit (kan annonsere features, de som ikke støtter paxos
bruker det ikke). TA med grovt hvor vi må endre (filer, kall),
litt om arkitekturen til ovs (datapath, odp, kernel modul osv), og ta med at vi bruker
ofctl til programmering istedenfor å bruke tid på POX.
Ta med at vi bruker odp for nå, men at vi har implementert (dvs dette kommer
i "implementation") i kernel også.}

\section{Paxos Flows}

\todo{flow regler, flyt osv for å få paxos til å funke.}
