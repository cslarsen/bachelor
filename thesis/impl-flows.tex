\section{Extending the OpenFlow Specification}

As discussed earlier in chapter \ref{chapter:openflow.design}, it would be
impractical to attempt to use existing actions in the OpenFlow specification
to implement the Paxos algorithm.
%
The OpenFlow specifications, as of version 1.4 \cite{openflow-1.4}, are
backward-compatible, meaning that a newer OpenFlow version will support all
features in older ones.  We will therefore choose to extend version 1.0
\cite{openflow-1.0}, because it was the first public version and therefore
the most widely supported.

The idea is to add a new \textit{Paxos action} with a parameters 
specifying whether to run the \textit{On Client}, \textit{On Accept} or 
\textit{On Learn} parts of the Paxos algorithm given in chapter
\vref{ch:simplifying.paxos}.
%
As shown in \vref{chapter:openflow.background}, we can then specify
precisely what kind of events we want to trigger Paxos ordering for and
combine that with other actions, such as which port the output should go to.

The part of the specification we need to extend is the 
\textit{Flow Action Structures} \cite[pp.~21--22]{openflow-1.0},
and it will be an \textit{optional} action \cite[pp.~3--6]{openflow-1.0}.
%
Listing \ref{listing:ofp10.action.type} shows the modification made to
the C\index{C} enumeration type \texttt{ofp10\_{}action\_{}type} from the
Open vSwitch source
code.\footnote{\texttt{ovs/include/openflow/openflow-1.0.h}}
We have included listing listing:ofp10.action.type as-is because this is how
it is defined in the published OpenFlow specification \cite{openflow-1.0}.
%
The listing is identical to the official specification, except for the
number suffix in \texttt{OFPAT10}.

\begin{lstlisting}[
  caption={Adding the \texttt{OFPAT10\_{}PAXOS} action to the OpenFlow
           specification},
  label={listing:ofp10.action.type}]
enum ofp10_action_type {
    OFPAT10_OUTPUT,             /* Output to switch port. */
    OFPAT10_SET_VLAN_VID,       /* Set the 802.1q VLAN id. */
    OFPAT10_SET_VLAN_PCP,       /* Set the 802.1q priority. */
    OFPAT10_STRIP_VLAN,         /* Strip the 802.1q header. */
    OFPAT10_SET_DL_SRC,         /* Ethernet source address. */
    OFPAT10_SET_DL_DST,         /* Ethernet destination address. */
    OFPAT10_SET_NW_SRC,         /* IP source address. */
    OFPAT10_SET_NW_DST,         /* IP destination address. */
    OFPAT10_SET_NW_TOS,         /* IP ToS (DSCP field, 6 bits). */
    OFPAT10_SET_TP_SRC,         /* TCP/UDP source port. */
    OFPAT10_SET_TP_DST,         /* TCP/UDP destination port. */
    OFPAT10_ENQUEUE,            /* Output to queue. */
    OFPAT10_PAXOS,              /* Extension: Run Paxos algorithm. */
    OFPAT10_VENDOR = 0xffff
};
\end{lstlisting}

The Paxos action has only one parameter: Which part of algorithm
\ref{ch:simplifying.paxos} to run.  The structure of this parameter is given
in listing \ref{listing:ofp10.action.paxos} and its its possible values are
defined in table \ref{table:paxos.event.codes}.
%
All action structures are required to start with the \texttt{type} and
\texttt{len} fields.

\begin{lstlisting}[
  caption={The \texttt{OFPAT10\_{}PAXOS} parameters},
  label={listing:ofp10.action.paxos}]
struct ofp10_action_paxos {
    ovs_be16 type;            /* Required: OFPAT10_PAXOS. */
    ovs_be16 len;             /* Required: Length is 8. */
    ovs_be32 paxos_event;
};
OFP_ASSERT(sizeof(struct ofp10_action_paxos) == 8);
\end{lstlisting}

As we can see, \texttt{paxos\_{}event} is encoded as a big-endian, unsigned
32-bit integer.
%
Its possible values are given in table \ref{table:paxos.event.codes}.

\begin{table}[H]
  \centering
  \begin{tabular}{|c|l|c|l|}
    \hline
      \textbf{Value} &
      \textbf{Meaning} &
      \textbf{Algorithm} &
      \textbf{\texttt{ovs-ofctl} argument}
      \\

    \hline
      \texttt{0x7A01} &
      Run ``On Accept'' &
      \ref{algorithm:paxos.simple.acceptor} &
      \texttt{paxos:onaccept}
      \\

    \hline
      \texttt{0x7A02} &
      Run ``On Learn'' &
      \ref{algorithm:paxos.simple.learner} &
      \texttt{paxos:onlearn}
      \\

    \hline
      \texttt{0x7A40} &
      Run ``On Client'' &
      \ref{algorithm:paxos.simple.client} &
      \texttt{paxos:onclient}
      \\

    \hline
  \end{tabular}
  \caption{Possible values for \texttt{paxos\_{}event} in listing
           \ref{listing:ofp10.action.paxos}}
  \label{table:paxos.event.codes}
\end{table}

The values in table \ref{table:paxos.event.codes} have been chosen
to correspond to the Ethernet types given
in table \vref{table:paxos.ethernet.type.encoding}, although they
could have been simply zero, one and two.
%
The last column contains the command-line arguments that will be
accepted by \texttt{ovs-ofctl} when adding flows.

Because of the thesis scope, we have only added a single action parameter
\texttt{paxos\_type}.
%
In a production environment, however, one would likely need several more.
%
For example, it could be useful to distinguish between different
\textit{sets} of Paxos nodes so they could operate independently of each
other on the same network.
%
Here, we have only \textit{one} set of Paxos nodes who all have the same
leader.

\subsection{Modifications to Open vSwitch}

As mentioned in section \vref{chapter:mininet}, the component in our system
that actually executes OpenFlow actions is \textit{Open vSwitch}.
To fully implement the new Paxos OpenFlow action, we need to do this in Open
vSwitch.  Details can be found in section \vref{chapter:compiling.ovs}.

Looking at table table \vref{table:paxos.event.codes}, the rightmost column
(\textbf{\texttt{ovs-ofctl} argument}) contains arguments to the Open
vSwitch command-line tool \texttt{ovs-ofctl}, that can be used to program
Paxos actions as smaller parts of bigger flows.

To demonstrate how elegantly one can set up flows that use Paxos ordering,
consider the below example for installing a flow on the switch
\texttt{S1}.

\begin{Verbatim}
sudo ovs-ofctl add-flow S1 \
               in_port=3,dl_type=0x7a40,actions=paxos:onclient,output:5
\end{Verbatim}

The above command installs a new flow entry on \texttt{S1}, matching packets
coming in on port 3 with the Ethernet type \texttt{0x7a40}.  
Referring to table \vref{table:paxos.event.codes}, we see that this flow
will match on packets of type \texttt{CLIENT}.

Furthermore, under \texttt{actions=}, we instruct Open vSwitch to run the
Paxos action with the parameter \texttt{onclient}.  This means that for
matching packets, Open vSwitch will dispatch the packet to the \textit{on
client} function (the argument \texttt{paxos:onclient}), described in
chapter \vref{chapter:paxos.client.message} and algorithm
\vref{algorithm:paxos.simple.client}.
%
This algorithm will output an accept message to output port 5
(\texttt{output:5})
If we want to explicitly set the destination address of the packet, one can
just prepend the output with the modification action
\texttt{mod\_dl\_dst=a1:b2:c3:d4:e5:f6}.
To send out on several ports, one just needs to add more \texttt{output:<N>}
actions, or the packet can be flooded on all ports with
\texttt{output:flood}.

What we are doing here is programming the switch's flow table using Paxos
primitives as constituent elements.  For the actual implementation, we refer
to the appendix, section \vref{chapter:compiling.ovs}.

We have implemented all of the actions in table
\ref{table:paxos.event.codes}, including multi-Paxos storage of packets in
slots, but with the important exception of the queue processing (algorithm
\ref{algorithm:paxos.simple.learner}, section
\ref{ch:simplifying.paxos}).

The above command translates client packets to Paxos \texttt{ACCEPT}
packets.  For a Paxos node on another switch, we can simply use the
\texttt{paxos:onaccept} action.  Since switch $S_2$ of figure
\vref{figure:paxos.on.switches} may receive Paxos messages from both $S_1$
and $S_3$, we may want to only react on packets that are explicitly
addressed to $S_2$.
%
To do so, assuming the MAC-address is \texttt{22:22:22:22:22:22}, one may
simply add a matching pattern for it, along with the obligatory check for
the Ethernet type field corresponding to an accept message
(\texttt{0x7A01}):

\begin{Verbatim}
sudo ovs-ofctl add-flow S2 \
    dl_src=11:11:11:11:11:11,\         # match from leader S1
    dl_dst=22:22:22:22:22:22,\         # match S2 MAC-address
    dl_type=0x7a01,\                   # match ACCEPT message
    actions=paxos:onaccept,\           # run "On Accept"
    mod_dl_src=22:22:22:22:22:22,\     # set source MAC address
    mod_dl_dst=33:33:33:33:33:33,\     # set destination MAC to S3
    output:5,\                         # output to port 5
    mod_dl_dst=11:11:11:11:11:11,\     # set destination MAC to S1
    output:1                           # output to port 1
\end{Verbatim}

The flow above is an \textit{actual} flow that we used---and verified to
work---in our network simulator.  If all conditions of algorithm
\ref{algorithm:paxos.simple.acceptor} are met, this flow wil send out a
learn message to $S_1$ and $S_3$.

Comparing this with writing equivalent flows as procedures in Python, this
is \textit{vastly} easier to do.  An \textit{excerpt} from the code for
accept-hanling in the Python controller is given below.

\begin{lstlisting}[
  caption={Shortened excerpt of Python code for handling Paxos accept messages},
  label={listing:python.accept}]
def on_accept(self, event, message):
  n, seqno, v = PaxosMessage.unpack_accept(message)
  src, dst = self.get_ether_addrs(event)

  # From leader?
  if src != self.leader.mac:
    return EventHalt # drop message

  slot = self.state.slots.get_slot(seqno)

  if n >= self.state.crnd and n != slot.vrnd:
    slot.vrnd = n
    slot.vval = v

    # Send learns to all
    for mac in self.state.ordered_nodes(self.mac):
      self.send_learn(mac, n, seqno, self.lookup_port(mac))

  return EventHalt
\end{lstlisting}

The code in the listing above is a shortened version of the
actual implementation.
%
Of course, we have had to actually \textit{implement} the above code in
equivalent C code in Open vSwitch, but the big gain is that the flows are
happening on the switch, and requires no upcall to the controller.
