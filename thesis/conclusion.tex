\chapter{Results and Conclusion}
\label{chapter:conclusion}

We set out to explore the merits of moving Paxos to the switch-level.
After having finished this thesis, we see that what we set out to do was a
gargantuan task.

It would be enough just to implement Paxos on the
controller, as it requires a lot of supporting software components like
booting the Mininet network, packing and unpacking of the Paxos IP-less
messages, the L2 Learning Switch (which also fully supports flows, without
any preconfiguration) for forwarding packets correctly, and finally to
attempt to do all of this on the large and complex Open vSwitch software,
running as a Linux kernel module.

\section{Paxos in the Controller}

In doing so, we implemented Paxos on a software controller, sending commands
to a switch.  Using this, we designed a distributed, replicated system with
guaranteed ordering for the UDP protocol.  The switches sent all packets up
to the controller, who then performed Paxos ordering before delivering
duplicated packets to end-hosts.

The main findings for this part of the thesis are:

\begin{itemize}
  \item Providing Paxos ordering transparently is possible, as we have
  demonstrated, and we were able to provide ordering of \acs{UDP} packets on
  a network consisting of three switches and nine end-hosts.

  Especially for UDP-based services, being able to guarantee same-sequence
  replication may be useful for certain types of services.


  \item There was some overhead in sending full packets to the controller
  and not using the flow tables for fast switching.  This was expected, but
  the \acs{RTT} was only around $1.4$ times larger than the equivalent
  test using flowlwss ICMP pings (section \ref{res:rtt.udp}).

  This result is not surprising, though, as packets need to perform
  additional link-hops to get from the switch to the controller and down
  again.

\end{itemize}

\section{TCP Replication}
\label{chapter:tcp.replication}

After verifying that UDP-replication worked and that the order of the
packets were correct, we attempted to move on to TCP.

We were able to duplicate TCP packets to all the hosts on the networks, but
as our system sent back replies to the client without ordering, we got a
race condition on the TCP packets:  The host with the shortest network path
to the client responded first back to the client, causing the client to send
a TCP FIN packet to it.  But because of replication, all hosts would receive
this, and they would thus close their connections before they had time to
serve the original client request.

While we were able to communicate with one client and one host,
the TCP replication was unreliable for the other hosts.

The TCP protocol's design runs contrary to the way we attempted to perform
replication:  It has an ordering scheme of its own, and seeks to establish
an end-to-end connection.

Being able to replicate TCP in this manner is most likely hard to get right,
and we chose not to investigate further.

\section{The Paxos Messages as a Network Protocol}

We designed a simple, non-IP protocol for exchanging Paxos-messages, putting
the parameters in the payload of Ethernet frames.  By definition, this is a
separate, non-IP protocol.

While the protocol was very simple, not accounting for retransmission,
data corruption and the like, it was demonstrated to work reliably on the
software simulator.

Furthermore, being so simple, it was very easy to reuse it when moving from
the Paxos controller down to Paxos in Open vSwitch.  Extracting Paxos
parameters was simply a matter of looking at given offsets in the packet
data.

\section{Paxos in the Switch}

We partially implemented Paxos on the switch itself by modifying the Open
vSwitch source code.  Due to complexities in the architecture of such an
advanced, production-grade software system, we were unable to fully test it.

While we implemented all parts of the simplified, multi-Paxos algorithm,
including handling of incoming client packets, accept and learn
messages, we were not able to send out stored packets on the network.
The sole reason was lack of time: This is complex software to understand,
and sometimes its architecture prevented us from making progress.

On the other hand, we were able to easily program flows using Paxos as
constituent primitives, and could combine them with existing OpenFlow
actions.  We believe this shows that programming not only the controller,
but new actions on the \textit{switch} as well can be very useful and even
make it easier to build complex networking flows.

