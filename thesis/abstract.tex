\begin{abstract}

\todo{Dette er bare et førsteutkast! Må forbedres.}

Sofware-defined networking (SDN) decouples the control plane from the
forwarding plane in switches, making it possible to create controllers in
software.
%
This has made it easier to build arbitrarily advanced networks, such as
networks that self-optimize for low power-consumption
\cite{Heller:2010:ESE:1855711.1855728}, cloud networks that route live
traffic to moving virtual machines \cite{erickson2008demonstration} as well
as making it possible to test out new networking protocols using simulators
on a laptop.

A trend in the computing industry is that more services are moved to the
Cloud, reaching a larger number of users.
%
Thus, it becomes more important that these services remain available despite
failure of individual machines, calling for distributed service replication.
%
A huge body of research in the distributed computing community have been
devoted to coming up with protocols that guarantee strong consistency among
the replicas of such a service.
%
The most cited paper on network resilience is the Paxos algorithm
\cite{Lam01} for message ordering.

By extending the OpenFlow protocol, we have implemented steady-phase Paxos
on a software switch, making it possible for controllers to compose
flows that leverage the message ordering guarantees of Paxos as a
constituent element.
%
As a demonstration of its use, we have built a system where UDP-based
services are replicated using Paxos transparently.

We conclude that the system shows promise and may be more efficient than
Paxos middleware software for certain network topologies.
%
While the system handles replication of UDP services, TCP may be harder to
support.

\todo{Konklusjon er ikke ferdig}


\end{abstract}
