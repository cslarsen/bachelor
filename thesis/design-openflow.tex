\section{OpenFlow}
\label{chapter:openflow.design}

Based on our previous discussion of network topology, we will here look at
how we can build such a system using OpenFlow.

We will start by looking at what capabilities OpenFlow can offer us, then
see how we can suit it to fit our needs.

\subsection{Capabilities in OpenFlow}
\label{chapter:details.openflow}

Now that we have discussed the main algorithms and our simplification of
them, we must take a look at what OpenFlow can offer us to reach our goals.

To see how we can enable Paxos functionality in OpenFlow, we need to take a
look at what features it can provide us.  There are several versions of the
OpenFlow specification, so we'll review the differences in each
one.\footnote{Unfortunately, the most widely supported version of OpenFlow in
simulators and controllers seem to be OpenFlow version 1.0.}

Naturally, we could implement the whole Paxos algorithm in the controller
itself.  Doing so should be quite trivial: One could modify an existing
implementation and make it use OpenFlow to transmit packets between
switches.  However, that would be very inefficient compared to running the
entire algorithm, or parts of it, in the switch.

In the tables below, you can see what version 1.0 offers in terms of core
functionality.  Some details have been omitted in favor of giving a clear
overview.  For details, see the full specification \cite{openflow-1.0}.

What's most important in 1.0, compared to later versions, is that it only
has \textit{one} flow table and only supports IPv4.  Other than that it has
counters per table, per flow, per port and per queue.  The headers that can
be used for matching packets are listed in table
\ref{table:openflow-1.0.headers} \vpageref{table:openflow-1.0.headers} and
the actions in table \ref{table:openflow-1.0.actions}
\vpageref{table:openflow-1.0.actions}.

By \textit{transport} address and port, we mean \ac{TCP} or \ac{UDP},
depending on what packet is currently matched\index{OpenFlow!transport}.

The specification requires compliant switches to update packet checksums
when modifying fields that require it.

\begin{table}
  \centering
  \begin{tabular}{l}
    \hline
     \textbf{Header-field} \\
    \hline
     Ingress port\index{OpenFlow!match on ingress port} \\

     Ethernet source address\index{OpenFlow!match on Ethernet} \\
     Ethernet destination address\index{OpenFlow!match on Ethernet} \\

     VLAN ID\index{OpenFlow!match on VLAN} \\
     VLAN priority\index{OpenFlow!match on VLAN} \\

     IP source address\index{OpenFlow!match on IP address} \\
     IP destination address\index{OpenFlow!match on IP address} \\
     IP protocol\index{OpenFlow!match on IP protocol} \\
     IP \ac{ToS} bits\index{OpenFlow!match on ToS} \\

     Transport source port\index{OpenFlow!UDP}\index{OpenFlow!TCP}\index{OpenFlow!transport} \\
     Transport destination port \\
    \hline
  \end{tabular}
  \caption{Header-fields that can be matched in OpenFlow 1.0.}
  \label{table:openflow-1.0.headers}
\end{table}
\index{OpenFlow!matching}
\index{OpenFlow!header-fields}
\index{OpenFlow!matching header-fields}

\begin{table}
  \centering
  \begin{tabular}{lll}
    \hline
      \textbf{Action} &
      \textbf{Required} &
      \textbf{Options} \\

    \hline
      Forward\index{OpenFlow!forwarding action} &
      Required &
               To all \\
     & & To controller \\
     & & To local switch \\
     & & To flow table \\
     & & To port \\
    \\
      Forward\index{OpenFlow!forwarding action} &
      Optional &
               Normal \\
     & & Flood\index{OpenFlow!flooding action} \\
    \\
      Enqueue\index{OpenFlow!enqueue action} &
      Optional &
      Can be used to implement, e.g., \acs{QoS}\index{OpenFlow!QoS} \\
    \\
      Drop\index{OpenFlow!drop action} &
      Required &
      Drop packet \\
    \\
      Modify-field\index{OpenFlow!modify-field action} &
      Optional &
               Set or replace VLAN ID\index{OpenFlow!modify VLAN} \\
     & & Set or replace VLAN priority\index{OpenFlow!modify VLAN} \\
     & & Strip any VLAN header\index{OpenFlow!modify VLAN} \\
     & & Replace Ethernet source address\index{OpenFlow!modify Ethernet addresses} \\
     & & Replace Ethernet destination address \\
     & & Replace IPv4 source address\index{OpenFlow!modify IPv4 addresses} \\
     & & Replace IPv4 destination address\index{OpenFlow!modify IPv4 addresses} \\
     & & Replace IPv4 \acs{ToS} bits\index{OpenFlow!modify ToS bits} \\
     & & Replace transport source port\index{OpenFlow!modify transport ports} \\
     & & Replace transport destination port\index{OpenFlow!modify transport ports} \\
    \hline
  \end{tabular}
  \caption{Actions in OpenFlow 1.0.}
  \label{table:openflow-1.0.actions}
\end{table}
\index{OpenFlow!actions}

\subsection{Limitations in OpenFlow}

By looking at what OpenFlow versions 1.1--1.3\index{OpenFlow!versions} offer,
one can see that we can't really make use of any of the added functionality
for running Paxos.  What we need is the ability to run programs on the
switch, which is something OpenFlow does not support at all.  Neither do
their action primitives add up to anything that could be used for
remembering state (such as the round and sequence number)  or executing
if-then-else statements.
%
We also need to store messages somewhere, so they can later be sent out in
the correct order.

One possible solution would be to insert a lot of flow table entries that
would match on specific round and sequence numbers.
%
But that would require a lot of flow entries, and would be neither an
elegant or practical solution.

OpenFlow does, however, offer us the ability to implement Paxos entirely
in the controller.
%
We also want to see if we can move parts of Paxos down to the switch itself.

To remember round and sequence number, we could possibly have used the
meta-data\index{OpenFlow!metadata} that is available in later versions of
OpenFlow.
%
However, metadata only exists as the packet is processed in the pipeline of
flow tables, and is erased when the packet actions are applied at the end.
%
The switch simply needs to store the state somewhere.

Mininet seems to support whatever version of OpenFlow that Open vSwitch uses,
as this is what it uses as a switch.  Open vSwitch\index{Open vSwitch}
supports OpenFlow versions 1.0---1.3\index{Open vSwitch!OpenFlow support}
almost fully, but support for 1.4 is flaky, and may crash.  So 1.4 is out of the question.

The most obvious component to look at is POX\index{POX}, our controller framework in
Python, which only supports OpenFlow 1.0.\footnote{It does seem to support
some \textit{Nicira extensions}\index{Nicira}\index{OpenFlow!Nicira
extensions}, though.  These are extensions that were originally added to
early OpenFlow versions, but much of it has been implemented in later
versions.  There is also a fork of POX (and other software projects) written
by CPqD that adds support for newer OpenFlow versions, but we haven't looked
at it.}

But the major point for our decision is what OpenFlow can offer us.
There simply is no way of executing general code, and there is no way to
remember state.

We have therefore implemented Paxos entirely on the controller, then
modified Open vSwitch to provide Paxos as a new, primitive OpenFlow action.

\subsection{Conclusion}

We have seen that OpenFlow does not seem to offer us the capabilities needed
to implement Paxos in the switch.

Thus, we have decided to create a prototype Paxos controller, then implement
it on the switch.
