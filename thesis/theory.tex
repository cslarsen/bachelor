\chapter{Theory}

\section{Paxos}
Paxos is a distributed consensus algorithm.\todo{Expand introduction and
details}.

Even though the basic Paxos algorithm in \cite{Lam01} is quite simple, it can be
a challenge to implement properly \cite{Chandra:2007:PML:1281100.1281103}.
\section{Software--defined networking}

\section{OpenFlow}
OpenFlow is a protocol that specifies how a switch and its controller can
communicate.

The controller can be a separately running program on a remote host. This
means that it can be arbitrarily complex and make decisions based on a
global overview of the networks it serves.

The controller program can be written in any language that implements the
OpenFlow specification, and the switches themselves can either be physical
or virtual.  In fact, many commercial network manufacturers now offer
OpenFlow--enabled switches.

By default, a switch will forward parts of a packet to the controller when
it does not know what to do with it.  The controller typically receives the
first few bytes of a packet---often along with a buffer ID pointing to the
complete packet in the switch---and can then decide what do to with it.  It
can instruct the switch to forward the packet, or it can add entries to the
switch's flow tables so that the switch will be able to handle specific
packets entirely or partly by itself in the future.  This initial passing of
packets to the controller incurs a performance hit, but as soon as the
switch has entries in its flow tables, processing is very fast.

\subsection{Flow Tables}

\section{Forth}

\section{OpenVSwitch}

