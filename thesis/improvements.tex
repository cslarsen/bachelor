\chapter{Improvements and future work}

Here we will discuss improvements that could have been made to our system
but was out of scope.

\section{Monitoring link--status}

OpenFlow makes it possible for controllers to receive notifications when
link status changes.  In OpenFlow 1.0, this is restricted to receiving
\textit{link up} and \textit{link down} notifications.

We could take advantage by monitoring the links to other switches.
If the leader goes down, leader--election should be performed.
If any switch goes down, then hosts should synchronize their state with
other hosts (and the same if a single host goes down).

\section{Leader--election}

Our system does not implement leader--election at all.  This should be part
of any complete Paxos system.

It would be easy to add more Paxos message types to our tables, and the
algorithms could be implemented as small bytecode programs.

\section{Using IP--fragmentation for buffering}

For the system to perform well, we don't want to store client packets in the
switch or the controller.  Instead, it would be nice if we could just pass
along client packets directly down to the end--hosts.

However, this means that the hosts will process the packets before we have a
chance to perform Paxos ordering.

We propose a neat solution to this problem.  When a switch receives a client
message, it will immediately perform IP--fragmentation of the message and
send the first fragment to the hosts.  The hosts networking stack will then
buffer the packet and wait for the last fragment.

When the Paxos consensus algorithm terminates, we will send the last
fragment down to the host, which will then pass the packet up the stack to
the server program.

We still need to store fragments, but if we choose the fragmentation offset
wisely, we need only store very small fragments.

The downside to this is that we break MTU rules, and some systems may behave
strange---or not at all.  But for our purposes we think this is a good
solution.

For this we need some new OpenFlow actions for fragmenting packets, storing
them and then forwarding the stored remaining fragment.

\section{Full Paxos support}

The most obvious improvement to this project would be to implement all of
Paxos: Trust, prepare and promise--messages and what has been mentioned
above.

However, we stated in the introduction that we wanted to constrain ourselves
to just look at how we could implement accept and learn--messages, as a
proof of concept.

Our solution would actually make it quite simple to add support for the full
Paxos algorithm.  Furthermore, our solution is very \textit{modular}, in the
sense that the controller can select which programs to compile to bytecode
and install on the switch.  This means we could have several different
leader election algorithms as small, self--contained programs, and select
which one to use for a given system.

The flow table rules are also extremely flexible as they are, and allows for
small, but easy, optimization opportunities (for instance, to support a new
Paxos message type only requires one new flow table entry and one new
program).

\todo{Flytt deler av dette ned til konklusjon, for det er et ganske sterkt
argument dette med modularisering og at det lett kan bygges videre på!}

\section{Security}

A networking thesis like this would not be complete without a brief
discussion of security concerns.

Basically, we have ignored all aspects of security.  While OpenFlow
communicates over a secure channel using \ac{TLS}, we allow bytecode to run
on the switches themselves.  Naturally, we should have chosen a bytecode
\acs{VM} that allows for \textit{sandboxed execution}.  I.e.,~it should not
be able to access any other facilities than pure programming constructs and
the restricted interface to the switch capabilities.

\todo{Tenk mere og skriv mere}

\section{Related work}

\todo{Skriv om andre prosjekter, er bla en load balancer skrevet for
  openflow og se på corfu--paper. Bør dette underkapittelet flyttes til
    introduksjon eller tidlig i oppgaven? Vet ikke hva som er vanlig
    struktur}

\section{Other platforms}

Look at other programmable networks.

Openflow 2.0 is rumoured to have an experimentation api, but the stuff in
this thesis is actually far ahead of that, more flexible.
It would actually allow one to experiment with what should be part of
OpenFlow 2.0.
