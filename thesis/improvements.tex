\chapter{Improvements and Future Work}

Here we present findings during design and implementation of switch-level
Paxos that suggest further study.

\section{Using the OpenFlow queue}

WHile the Paxos leader decides the order of each packet, it is the learner
that actually makes sure that they are processed in the correct sequence.
%
In multi-Paxos, the learner has essentially a \textit{queue} of packets to
be sent.  As soon as it has a contiguous sequence of packets, starting from
the last processed one, it will process them.

OpenFlow has a queueing mechanism, which can be used to implement \ac{QoS}.
It would be very interesting to investigate whether this could be used for
multi-Paxos slots.  We have not looked more into this.

\section{Monitoring Link Status}

OpenFlow makes it possible for controllers to receive notifications when
link-status\index{link-status} changes.  In OpenFlow 1.0, this is restricted
to receiving \textit{link up} and \textit{link down}
notifications\index{OpenFlow!link-status}.

We could take advantage by monitoring the links to other switches,
triggering, for instance, leader election when needed.

\section{Full Paxos Support}

The most obvious improvement to this project would be to implement all of
Paxos: Trust\index{Paxos!trust}, prepare\index{Paxos!prepare} and
promise messages\index{Paxos!promise} and what has been mentioned above.

We clearly stated the scope of exploration in chapter
\ref{chapter:introduction}, restricting ourselves to only handling accept
and learn messages.  At the same time, it should be possible to build upon
our implementation to support full Paxos.  We have left several placeholders
for code to do so, and adding new message types is trivial.
