\chapter{OpenFlow}
\section{Features in OpenFlow versions 1.0--1.4}

To see how we can enable Paxos functionality in OpenFlow, we need to take a
look at what features it can provide us.

Naturally, we could implement the whole Paxos algorithm in the controller
itself.  Doing so should be quite trivial: One could take an existing
implementation and make some slight modifications to it.  There is in fact
some merit to this idea, because we could deploy systems where most of the
Paxos algorithm is implemented on the controllers, leaving some minor Paxos
code to run on the end--systems (for instance, the end--systems would need
to recognize Paxos LEARN--messages and extract the payload).

However, that would be very inefficient compared to running the entire
algorithm---or parts of it---on the switch.

OpenFlow controllers are meant to set up entries in the flow tables as they
see fit.  In an efficient OpenFlow--system, the
controllers will initially get lots of packets and program the flow tables.
Over time, more work can be performed by the use of flow tables, and so the
controllers will only update the flow tables intermittently.

The flow tables themselves will then be able to match given
packets to the specification and perform actions on them.  These flow tables
are {\em designed} to allow very efficient implementations in hardware.  In
other words, this means that the actions that can be performed are very
simple and far from a fully--fledged programmable system.

Because of this, we need to take a look at what operations we can perform in
OpenFlow.  The versions vary quite a bit on this point, and a lot of
software only support given versions.  So let's get an overview of what is
possible in the different OpenFlow versions.




\section{Extending OpenFlow}

Based on our previous discussions on OpenFlow functionality and our
simplified Paxos algorithm, we now present some new primitives for OpenFlow
that can be used to implement \texttt{ACCEPT} and \texttt{LEARN} messages.

\subsection{Data table}

First we need to be able to store the round number (\texttt{rnd}) and a
packet ID (\texttt{vval}).

We propose that the switch adds a new table that can be used to store
values.  To conserve memory, we will allow 256 entries in the table, each
with a field of 32 bits, totalling a mere 1024 bytes.

The table is meant to be used by the switch itself and not by the
controller, but one could propose store and retrieve operations for the
controller.  These are, however, not needed for our purposes.

There should only be need for one table per switch.  This is in contrast to
the OpenFlow counters, which are per flow table, per flow entry and so on.

\subsection{Table read and write actions}

The switch will need to be able to read and write to the data table.
These will be available as actions in the flow table entries.

At the very least, we will need actions to perform the following operations.

\todo{Would it be better with a Forth--like language and a small stack VM
  instead?}

\begin{table}[H]
  \begin{tabular}{|l|l|l|}
    \hline \textbf{Name} & \textbf{Parameters} & \textbf{Description} \\
    \hline GET & 8--bit table offset & Returns value from table \\
    \hline SET & 8--bit table offset and 32--bit value & Sets value in table \\
    \hline INC & 8--bit table offset & Increments value at given table location \\
    \hline CMP & ... & ... \\
    \hline
  \end{tabular}
  \caption{Data table operations}
  \label{extended.openflow.operations}
\end{table}

\begin{verbatim}
What we need is actually:

- to MATCH a given packet that contains special data, e.g. (ACCEPT, n, v),
  and match only "IF TABLE[0] >= n"
- A resulting action would then be to "SET TABLE[0], n" and then "FLOOD
  (LEARN, n, v)" to the group "switches".
- If we're to send the packets immediately to the end--systems, we need to
  be able to FRAGMENT packets, store some IDs and then send the remaining
  fragment.  For simplicity, perhaps I should wait with this part?
- I think we should look at the complete Paxos algorithm and see what would
  be needed to implement everything.
- Note also how the table ops almost look like registers
- Also note how we are almost creating a Turing complete system. That's what
  we want to avoid, but for generality, the is probably what is required.
\end{verbatim}

\subsection{New matching patterns}
