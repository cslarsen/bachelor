\chapter{OpenFlow}
\section{Capabilities in OpenFlow}
\label{chapter:details.openflow}

\todo{Show ONLY openflow 1.3, possibly 1.4}

To see how we can enable Paxos functionality in OpenFlow, we need to take a
look at what features it can provide us.  There are several versions of the
OpenFlow specification, so we'll review the differences in each
one.\footnote{Unfortunately, the most widely supported version of OpenFlow in
simulators and controllers seem to be OpenFlow version 1.0.}

Naturally, we could implement the whole Paxos algorithm in the controller
itself.  Doing so should be quite trivial: One could modify an existing
implementation and make it use OpenFlow to transmit packages between
switches.  However, that would be very inefficient compared to running the
entire algorithm, or parts of it, on the switch.

Prior to OpenFlow 1.0 \cite{openflow-1.0} there were several working
drafts not meant for implementation.  We will not look at these prior
versions.

In the tables below, you can see what version 1.0 offers in terms of core
functionality.  Some details have been omitted in favour of giving a clear
overview.  For details, see the full specification \cite{openflow-1.0}.

What's most important in 1.0, compared to later versions, is that it only
has \textit{one} flow table and only supports IPv4.  Other than that it has
counters per table, per flow, per port and per queue.  The headers that can
be used for matching packets are listed in table
\ref{table:openflow-1.0.headers} \vpageref{table:openflow-1.0.headers} and
the actions in table \ref{table:openflow-1.0.actions}
\vpageref{table:openflow-1.0.actions}.

By \textit{transport} address and port, we mean \acs{TCP} or \acs{UDP},
depending on what packet is currently matched.

The specification requires compliant switches to update packet checksums
when modifying fields that require it.

\begin{table}
  \centering
  \begin{tabular}{l}
     \textbf{Header--field} \\
    \hline
     Ingress port \\

     Ethernet source address \\
     Ethernet destination address \\

     VLAN ID \\
     VLAN priority \\

     IP source address \\
     IP destination address \\
     IP protocol \\
     IP \ac{ToS} bits \\

     Transport source port \\
     Transport destination port \\
  \end{tabular}
  \caption{Header--fields that can be matched in OpenFlow 1.0.}
  \label{table:openflow-1.0.headers}
\end{table}

\begin{table}
  \centering
  \begin{tabular}{lll}
      \textbf{Action} &
      \textbf{Required} &
      \textbf{Options} \\

    \hline
      Forward &
      Required &
               To all \\
     & & To controller \\
     & & To local switch \\
     & & To flow table \\
     & & To port \\
    \\
      Forward &
      Optional &
               Normal \\
     & & Flood \\
    \\
      Enqueue &
      Optional &
      Can be used to implement, e.g., \acs{QoS} \\
    \\
      Drop &
      Required &
      Drop packet \\
    \\
      Modify--field &
      Optional &
               Set or replace VLAN ID \\
     & & Set or replace VLAN priority \\
     & & Strip any VLAN header \\
     & & Replace Ethernet source address \\
     & & Replace Ethernet destination address \\
     & & Replace IPv4 source address \\
     & & Replace IPv4 destination address \\
     & & Replace IPv4 \acs{ToS} bits \\
     & & Replace transport source port \\
     & & Replace transport destination port \\
  \end{tabular}
  \caption{Actions in OpenFlow 1.0.}
  \label{table:openflow-1.0.actions}
\end{table}

\todo{Skriv litt mer om features i openflow--versjoner nyere enn 1.0, ikke
  lag tabeller, bare list opp store forskjeller. Bruk dette i
    argumentasjonen under.}


\section{Extending OpenFlow}

Based on our previous discussions on OpenFlow functionality and our
simplified Paxos algorithm, we now present some new primitives for OpenFlow
that can be used to implement \texttt{ACCEPT} and \texttt{LEARN} messages.

\subsection{Data table}

First we need to be able to store the round number (\texttt{rnd}) and a
packet ID (\texttt{vval}).

We propose that the switch adds a new table that can be used to store
values.  To conserve memory, we will allow 256 entries in the table, each
with a field of 32 bits, totalling a mere 1024 bytes.

The table is meant to be used by the switch itself and not by the
controller, but one could propose store and retrieve operations for the
controller.  These are, however, not needed for our purposes.

There should only be need for one table per switch.  This is in contrast to
the OpenFlow counters, which are per flow table, per flow entry and so on.

\subsection{Table read and write actions}

The switch will need to be able to read and write to the data table.
These will be available as actions in the flow table entries.

At the very least, we will need actions to perform the following operations.

\todo{Would it be better with a Forth--like language and a small stack VM
  instead?}

\begin{table}[H]
  \begin{tabular}{|l|l|l|}
    \hline \textbf{Name} & \textbf{Parameters} & \textbf{Description} \\
    \hline GET & 8--bit table offset & Returns value from table \\
    \hline SET & 8--bit table offset and 32--bit value & Sets value in table \\
    \hline INC & 8--bit table offset & Increments value at given table location \\
    \hline CMP & ... & ... \\
    \hline
  \end{tabular}
  \caption{Data table operations}
  \label{extended.openflow.operations}
\end{table}

\begin{verbatim}
What we need is actually:

- to MATCH a given packet that contains special data, e.g. (ACCEPT, n, v),
  and match only "IF TABLE[0] >= n"
- A resulting action would then be to "SET TABLE[0], n" and then "FLOOD
  (LEARN, n, v)" to the group "switches".
- If we're to send the packets immediately to the end--systems, we need to
  be able to FRAGMENT packets, store some IDs and then send the remaining
  fragment.  For simplicity, perhaps I should wait with this part?
- I think we should look at the complete Paxos algorithm and see what would
  be needed to implement everything.
- Note also how the table ops almost look like registers
- Also note how we are almost creating a Turing complete system. That's what
  we want to avoid, but for generality, the is probably what is required.
\end{verbatim}

\subsection{New matching patterns}
