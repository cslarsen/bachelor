\chapter{Implementation}
\label{chapter:implementation}

Based on the design in chapter \ref{chapter:design}, we will now look at
implementation details.

To be able to run Paxos in the switch, we must first extend the OpenFlow
Switch Specification with a new \textit{Paxos action}.
\index{flows!Paxos action}\index{OpenFlow!Paxos action}%
\index{Paxos!OpenFlow action}%
%
This will allow us to freely \textit{compose} flows that run the
Paxos algorithm as one part of their actions.
%
Finally, we must modify Open vSwitch so that we can run the new action.

\section{Extending the OpenFlow Specification}

As discussed earlier in chapter \ref{chapter:openflow.design}, it would be
impractical to attempt to use existing actions in the OpenFlow specification
to implement the Paxos algorithm.
%
The OpenFlow specifications, as of version 1.4 \cite{openflow-1.4}, are
backward-compatible, meaning that a newer OpenFlow version will support all
features in older ones.  We will therefore choose to extend version 1.0
\cite{openflow-1.0}, because it was the first public version and therefore
the most widely supported.

The idea is to add a new \textit{Paxos action} with a parameters 
specifying whether to run the \textit{On Client}, \textit{On Accept} or 
\textit{On Learn} parts of the Paxos algorithm given in chapter
\vref{ch:simplifying.paxos}.
%
As shown in \vref{chapter:openflow.background}, we can then specify
precisely what kind of events we want to trigger Paxos ordering for and
combine that with other actions, such as which port the output should go to.

The part of the specification we need to extend is the 
\textit{Flow Action Structures} \cite[pp.~21--22]{openflow-1.0},
and it will be an \textit{optional} action \cite[pp.~3--6]{openflow-1.0}.
%
Listing \ref{listing:ofp10.action.type} shows the modification made to
the C\index{C} enumeration type \texttt{ofp10\_{}action\_{}type} from the
Open vSwitch source
code.\footnote{\texttt{ovs/include/openflow/openflow-1.0.h}}
We have included listing listing:ofp10.action.type as-is because this is how
it is defined in the published OpenFlow specification \cite{openflow-1.0}.
%
The listing is identical to the official specification, except for the
number suffix in \texttt{OFPAT10}.

\begin{lstlisting}[
  caption={Adding the \texttt{OFPAT10\_{}PAXOS} action to the OpenFlow
           specification},
  label={listing:ofp10.action.type}]
enum ofp10_action_type {
    OFPAT10_OUTPUT,             /* Output to switch port. */
    OFPAT10_SET_VLAN_VID,       /* Set the 802.1q VLAN id. */
    OFPAT10_SET_VLAN_PCP,       /* Set the 802.1q priority. */
    OFPAT10_STRIP_VLAN,         /* Strip the 802.1q header. */
    OFPAT10_SET_DL_SRC,         /* Ethernet source address. */
    OFPAT10_SET_DL_DST,         /* Ethernet destination address. */
    OFPAT10_SET_NW_SRC,         /* IP source address. */
    OFPAT10_SET_NW_DST,         /* IP destination address. */
    OFPAT10_SET_NW_TOS,         /* IP ToS (DSCP field, 6 bits). */
    OFPAT10_SET_TP_SRC,         /* TCP/UDP source port. */
    OFPAT10_SET_TP_DST,         /* TCP/UDP destination port. */
    OFPAT10_ENQUEUE,            /* Output to queue. */
    OFPAT10_PAXOS,              /* Extension: Run Paxos algorithm. */
    OFPAT10_VENDOR = 0xffff
};
\end{lstlisting}

The Paxos action has only one parameter: Which part of algorithm
\ref{ch:simplifying.paxos} to run.  The structure of this parameter is given
in listing \ref{listing:ofp10.action.paxos} and its its possible values are
defined in table \ref{table:paxos.event.codes}.
%
All action structures are required to start with the \texttt{type} and
\texttt{len} fields.

\begin{lstlisting}[
  caption={The \texttt{OFPAT10\_{}PAXOS} parameters},
  label={listing:ofp10.action.paxos}]
struct ofp10_action_paxos {
    ovs_be16 type;            /* Required: OFPAT10_PAXOS. */
    ovs_be16 len;             /* Required: Length is 8. */
    ovs_be32 paxos_event;
};
OFP_ASSERT(sizeof(struct ofp10_action_paxos) == 8);
\end{lstlisting}

As we can see, \texttt{paxos\_{}event} is encoded as a big-endian, unsigned
32-bit integer.
%
Its possible values are given in table \ref{table:paxos.event.codes}.

\begin{table}[H]
  \centering
  \begin{tabular}{|c|l|c|l|}
    \hline
      \textbf{Value} &
      \textbf{Meaning} &
      \textbf{Algorithm} &
      \textbf{\texttt{ovs-ofctl} argument}
      \\

    \hline
      \texttt{0x7A01} &
      Run ``On Accept'' &
      \ref{algorithm:paxos.simple.acceptor} &
      \texttt{paxos:onaccept}
      \\

    \hline
      \texttt{0x7A02} &
      Run ``On Learn'' &
      \ref{algorithm:paxos.simple.learner} &
      \texttt{paxos:onlearn}
      \\

    \hline
      \texttt{0x7A40} &
      Run ``On Client'' &
      \ref{algorithm:paxos.simple.client} &
      \texttt{paxos:onclient}
      \\

    \hline
  \end{tabular}
  \caption{Possible values for \texttt{paxos\_{}event} in listing
           \ref{listing:ofp10.action.paxos}}
  \label{table:paxos.event.codes}
\end{table}

The values in table \ref{table:paxos.event.codes} have been chosen
to correspond to the Ethernet types given
in table \vref{table:paxos.ethernet.type.encoding}, although they
could have been simply zero, one and two.
%
The last column contains the command-line arguments that will be
accepted by \texttt{ovs-ofctl} when adding flows.

Because of the thesis scope, we have only added a single action parameter
\texttt{paxos\_type}.
%
In a production environment, however, one would likely need several more.
%
For example, it could be useful to distinguish between different
\textit{sets} of Paxos nodes so they could operate independently of each
other on the same network.
%
Here, we have only \textit{one} set of Paxos nodes who all have the same
leader.

\subsection{Modifications to Open vSwitch}

As mentioned in section \vref{chapter:mininet}, the component in our system
that actually executes OpenFlow actions is \textit{Open vSwitch}.
To fully implement the new Paxos OpenFlow action, we need to do this in Open
vSwitch.  Details can be found in section \vref{chapter:compiling.ovs}.

Looking at table table \vref{table:paxos.event.codes}, the rightmost column
(\textbf{\texttt{ovs-ofctl} argument}) contains arguments to the Open
vSwitch command-line tool \texttt{ovs-ofctl}, that can be used to program
Paxos actions as smaller parts of bigger flows.

To demonstrate how elegantly one can set up flows that use Paxos ordering,
consider the below example for installing a flow on the switch
\texttt{S1}.

\begin{Verbatim}
sudo ovs-ofctl add-flow S1 \
               in_port=3,dl_type=0x7a40,actions=paxos:onclient,output:5
\end{Verbatim}

The above command installs a new flow entry on \texttt{S1}, matching packets
coming in on port 3 with the Ethernet type \texttt{0x7a40}.  
Referring to table \vref{table:paxos.event.codes}, we see that this flow
will match on packets of type \texttt{CLIENT}.

Furthermore, under \texttt{actions=}, we instruct Open vSwitch to run the
Paxos action with the parameter \texttt{onclient}.  This means that for
matching packets, Open vSwitch will dispatch the packet to the \textit{on
client} function (the argument \texttt{paxos:onclient}), described in
chapter \vref{chapter:paxos.client.message} and algorithm
\vref{algorithm:paxos.simple.client}.
%
This algorithm will output an accept message to output port 5
(\texttt{output:5})
If we want to explicitly set the destination address of the packet, one can
just prepend the output with the modification action
\texttt{mod\_dl\_dst=a1:b2:c3:d4:e5:f6}.
To send out on several ports, one just needs to add more \texttt{output:<N>}
actions, or the packet can be flooded on all ports with
\texttt{output:flood}.

What we are doing here is programming the switch's flow table using Paxos
primitives as constituent elements.  For the actual implementation, we refer
to the appendix, section \vref{chapter:compiling.ovs}.

We have implemented all of the actions in table
\ref{table:paxos.event.codes}, including multi-Paxos storage of packets in
slots, but with the important exception of the queue processing (algorithm
\ref{algorithm:paxos.simple.learner}, section
\ref{ch:simplifying.paxos}).

The above command translates client packets to Paxos \texttt{ACCEPT}
packets.  For a Paxos node on another switch, we can simply use the
\texttt{paxos:onaccept} action.  Since switch $S_2$ of figure
\vref{figure:paxos.on.switches} may receive Paxos messages from both $S_1$
and $S_3$, we may want to only react on packets that are explicitly
addressed to $S_2$.
%
To do so, assuming the MAC-address is \texttt{22:22:22:22:22:22}, one may
simply add a matching pattern for it, along with the obligatory check for
the Ethernet type field corresponding to an accept message
(\texttt{0x7A01}):

\begin{Verbatim}
sudo ovs-ofctl add-flow S2 \
    dl_src=11:11:11:11:11:11,\         # match from leader S1
    dl_dst=22:22:22:22:22:22,\         # match S2 MAC-address
    dl_type=0x7a01,\                   # match ACCEPT message
    actions=paxos:onaccept,\           # run "On Accept"
    mod_dl_src=22:22:22:22:22:22,\     # set source MAC address
    mod_dl_dst=33:33:33:33:33:33,\     # set destination MAC to S3
    output:5,\                         # output to port 5
    mod_dl_dst=11:11:11:11:11:11,\     # set destination MAC to S1
    output:1                           # output to port 1
\end{Verbatim}

The flow above is an \textit{actual} flow that we used---and verified to
work---in our network simulator.  If all conditions of algorithm
\ref{algorithm:paxos.simple.acceptor} are met, this flow wil send out a
learn message to $S_1$ and $S_3$.

Comparing this with writing equivalent flows as procedures in Python, this
is \textit{vastly} easier to do.  An \textit{excerpt} from the code for
accept-hanling in the Python controller is given below.

\begin{lstlisting}[
  caption={Shortened excerpt of Python code for handling Paxos accept messages},
  label={listing:python.accept}]
def on_accept(self, event, message):
  n, seqno, v = PaxosMessage.unpack_accept(message)
  src, dst = self.get_ether_addrs(event)

  # From leader?
  if src != self.leader.mac:
    return EventHalt # drop message

  slot = self.state.slots.get_slot(seqno)

  if n >= self.state.crnd and n != slot.vrnd:
    slot.vrnd = n
    slot.vval = v

    # Send learns to all
    for mac in self.state.ordered_nodes(self.mac):
      self.send_learn(mac, n, seqno, self.lookup_port(mac))

  return EventHalt
\end{lstlisting}

The code in the listing above is a shortened version of the
actual implementation.
%
Of course, we have had to actually \textit{implement} the above code in
equivalent C code in Open vSwitch, but the big gain is that the flows are
happening on the switch, and requires no upcall to the controller.


As discussed in \vref{chapter:theory.flow.table}, well-designed controllers
should install flows incrementally as they learn the network topology.
%
We must therefore first implement a system that works entirely without flow
entries. (Dette er en del av design-diskusjon, vi skal bare implementere det
her).

Next, we will implement flows in the system. As discussed previously, this
requires an extension to the OpenFlow-protocol and the switch software we
use, Open vSwitch.

Mer tekst: At vi har, i designet, extenda OpenFlow + Open vSwitch slik at vi
kan kjøre kode. Vi viser implementasjonen her, husk å skille på design og
implementasjon klart og tydelig.

\label{implementation.simplified.paxos}

We will implement algorithms \ref{algorithm:paxos.simple.acceptor} 
and \ref{algorithm:paxos.simple.learner} in a combination of OpenFlow
matches\index{OpenFlow!matching} and its extensions that were introduced
in \vref{chapter:extending.openflow}.

\section{An L2 learning switch in OpenFlow}

When you write an OpenFlow controller, the flow table is empty and all
packets will by default be delivered to the controller.

The controller must then decide what to do with the packets.  If we don't
implement any sort of forwarding behvaiour for the packets, none of the
hosts will be able to communicate.

So our system will need a forwarding mechanism at the bottom of the Paxos
networking capabilities.  The simplest system is just to implement a
\textit{hub}:  For each packet coming in to the switch, flood it (or,
\textit{rebroadcast}) to all ports, and let each connected host decide 
to receive packets meant for them (algorithm \ref{algorithm:l2.hub}).

\begin{algorithm}
  \begin{algorithmic}
    \On{Ethernet packet $e$}{port $p$}
      \State \textbf{flood} $e$ \Comment{Send packet out on \textit{all} ports}
    \EndOn
  \end{algorithmic}
  \caption{An L2 hub algorithm}
  \label{algorithm:l2.hub}
\end{algorithm}

A slightly better approach is to implement an \ac{L2} learning switch.
The difference from the flood--to--all hub above is that we create a table
that maps MAC--addresses to ports and then forward each packet to a single
port.  We then achieve less traffic on the network.

As we build up this table we could also install flow table entries so that the
switch will be able to forward packets by itself.  This is indicated in
algorithm \vref{algorithm:l2.learning.switch} with
\textbf{add.flowtable.entry}, but is an optional step.

\begin{algorithm}
  \begin{algorithmic}
    \State $M \gets \emptyset$\Comment{Set of $\langle address,\ port \rangle$--tuples}
    \State
    \On{Ethernet packet $e$}{port $p$}
      \State $M \gets M \cup \langle e_{src},\ p \rangle$ \Comment{Learn
        port $p$ for $e_{src}$ (source MAC--address)}
      \State
      \State \textbf{add OpenFlow rule }(for Ethernet packets to
        $e_{src}$, forward to port $p$)
      \State
      \If{$\{ \exists q : \langle e_{dst},\ q \rangle \in M \}$}
        \Comment{See if we know the destination port $q$}
        \State \textbf{forward} $e$ \textbf{to} port $q$ for $e_{dst}$ in $M$
      \Else
        \State \textbf{flood} $e$ \Comment{Act as hub; rebroadcast packet
          $e$ to all ports}
      \EndIf
    \EndOn
  \end{algorithmic}
  \caption{An L2 learning switch algorithm for an OpenFlow controller}
  \label{algorithm:l2.learning.switch}
\end{algorithm}
\todo{Fiks membership test sjekk her, skal en feks bruke $\exists$ eller noe
  sånt, og er wildcard--operator bruk korrekt?}

As you can see, algorithm \ref{algorithm:l2.learning.switch} will need to
run at least twice before it will know both the source and destination ports
for two MAC--addresses.  If we send an \textit{\acs{ICMP} ping packet} from
host $a$ to $b$, the switch running the algorithm will first learn which port
$a$ is on, and then flood the packet out all ports (it doesn't know which
port $b$ is on, yet).

$b$ will then receive the packet\footnote{The other hosts' networking stack
will simply drop the packet, as it's not for them---unless their \acs{NIC} is
running in \textit{promiscuous mode}, capturing all packets.} and send an \acs{ICMP}
ping reply packet.  When this reaches the switch, it will learn which port
$b$ is on and is now able to do a packet forwarding instead of a flood,
because it knows which port $a$ is on (the packet from $b$ has $b_{address}$ as
source MAC--address and $a_{address}$ as destination MAC--address).

We can also install rules in the OpenFlow flow table so that
subsequent packets to these two hosts will be forwarded automatically to
their respective port---without any interaction from the controller.  It
also means that the controller will not see those packets anymore.  As
mentioned elsewhere, each flow table entry has an associated set of idle and
hard timeout counters.  We've not indicated values for these here, but
typically one sets the idle timeout to 10 seconds and the hard timeout to 60
seconds.  This means that we have to keep adding the rules again and again,
but from now that will be done automatically by the algorithm.\footnote{The
timeouts help keep the flow table from going full.}

Finally, one must realize that it doesn't matter if the ports are connected
\textit{directly} to hosts with the associated MAC--addresses.  Even if the
ports are links to other networks, we know that a MAC--address has been seen
coming from this port, and should therefore be reachable, somehow, on that
port.

This algorithm has been implemented in Python using the POX controller, with
the exception that we don't install forwarding flow table
entries.\footnote{This is just for implementation simplicity, since we also
need to install flows that react on PAXOS messages.  In a production
system, we would install L2 forwarding with a lower priority than those
entries reacting on PAXOS--messages, otherwise the PAXOS handling code would
never be run.}

Our algorithm is a well--known implementation strategy for learning
switches.  Ours is based on the one given in the OpenFlow tutorial \todo{cite!}.
There are additional checks that we don't perform, such as
asserting that the source and destination ports are not the same when
installing flow entries (see, e.g., the pseudo--code in figure 3 of
\cite{Canini:2012:NWT:2228298.2228312}).


\section{Paxos Message Wire Format}

When exchanging Paxos messages between switches, we need a way to identify
them.
%
A well-known use of OpenFlow is to create entirely new, non-IP protocols
by matching on fields in the Ethernet header\index{Ethernet!header}
\cite[Example 4, p.~73]{McKeown:2008:OEI:1355734.1355746}.
%
We will tag Paxos messages with special values in the \textit{Ethernet
  type}-field\index{Ethernet!type}.
%
This field is two octets wide (i.e.,~16 bits), so we can use the most
significant one to mark packets as carrying Paxos messages, and the
least significant one for the kind of Paxos message (table
\ref{table:paxos.ethernet.type.encoding}).

\begin{table}[H]
  \centering
  \begin{tabular}{l|c|c|}
    \cline{2-3}
      & \multicolumn{2}{c|}{\textbf{Ethernet Type Field}} \\
      & \multicolumn{2}{c|}{16 bits} \\

    \hline
      \multicolumn{1}{|l|}{\textbf{Message Type}} &
      \textbf{Most Significant} &
      \textbf{Least Significant} \\

    \hline
      \multicolumn{1}{|l|}{\texttt{PAXOS JOIN}} &
      \texttt{0x7A} &
      \texttt{0x00} \\

    \hline
      \multicolumn{1}{|l|}{\texttt{PAXOS ACCEPT}} &
      \texttt{0x7A} &
      \texttt{0x01} \\

    \hline
      \multicolumn{1}{|l|}{\texttt{PAXOS LEARN}} &
      \texttt{0x7A} &
      \texttt{0x02} \\

    \hline
  \end{tabular}
  \caption{Encoding of \texttt{PAXOS} messages in the \textit{Ethernet
    type} field.}
  \label{table:paxos.ethernet.type.encoding}
\end{table}

There is no particular reason for the specific values used in table
\ref{table:paxos.ethernet.type.encoding}, but since \texttt{ACCEPT}
and \texttt{LEARN} messages share the first parameters, they
could be bits that could both be turned on to send a combined
\texttt{ACCEPT-and-LEARN} message.  If both bits are zero, it becomes
a \texttt{JOIN} message.
%
We cannot use values below \texttt{0x600}, because that is used by
Ethernet to signify payload size.

Using the Ethernet type for identifying Paxos messages makes it very
convenient to match the different messages in OpenFlow's flow
tables\index{OpenFlow!flow table}.

We now have to define the payload structure for Paxos messages.
Table \ref{table:paxos.ethernet.packet} defines the parameters
each message type will contain.
%
It will consist of consecutive 32-bit values for storing parameters,
followed by the a full client packet in \texttt{ACCEPT} messages.
%
Each type of message will trigger the corresponding algorithms in 
\vref{ch:simplifying.paxos}.  The \texttt{JOIN} message is discussed in
chapter \ref{chapter:paxos.join.message}.

\begin{table}[H]
  \centering
  \begin{tabular}{l|l|c|c|c|}
    \hline
      \multirow{2}{*}{\dots} &
      \multicolumn{1}{c|}{\textbf{Ethernet Type}} &
      \multicolumn{2}{c|}{\textbf{Parameters}} &
      \textbf{Payload} \\

      &
      \multicolumn{1}{c|}{16 bits} &
      \multicolumn{1}{c}{32 bits} &
      \multicolumn{1}{c|}{32 bits} &
      \dots \\

    \hline
      \dots & \texttt{PAXOS JOIN}   & $node_{id}$ & MAC source &
        \multicolumn{1}{c}{} \\

    \hline
      \dots & \texttt{PAXOS CLIENT} & \textit{ignored} & \textit{ignored} &
          $v$ (client packet) \\

    \hline
      \dots & \texttt{PAXOS ACCEPT} & $n$ (round) & $seq$ (sequence) &
          $v$ (client packet) \\

    \hline
      \dots & \texttt{PAXOS LEARN}  & $n$ (round) & $seq$ (sequence) &
          \multicolumn{1}{c}{} \\

    \cline{1-4}
  \end{tabular}

  \caption{The structure of \acs{L2} Paxos messages.  Not shown her is
           the preceding Ethernet fields.}
  \label{table:paxos.ethernet.packet}
\end{table}
\index{Paxos!message structure}

At this point we should discuss what will happen when the round or sequence
number reaches the maximum number possible.
%
A good solution would be to program the Paxos nodes to allow values to
roll around to zero when passing the maximum value of $2^{31}-1$, so that
we would never run out of numbers.
%
This is a detail that is irrelevant for our stated goals, but a complete
implementation should naturally allow for infinite sequences.

\subsection{The \texttt{PAXOS ACCEPT} Message}
\label{chapter:paxos.accept.message}

The \texttt{ACCEPT} message contains the round and sequence numbers for the
embedded client packet.  They correspond to the variables $n$, $seq$ and
$v$ of the Paxos algorithms in chapter \vref{ch:simplifying.paxos},
respectively.

It will start algorithm \ref{algorithm:paxos.simple.acceptor} and send out
\texttt{LEARN} messages, if the conditions are right.

Since it shares the first parameters with the \texttt{LEARN} message, and
since only the leader send them out, a triggering ofshare the first parameters with the 
The \texttt{ACCEPT} message share the first parameters with the
\texttt{LEARN} message.
%


\subsection{The \texttt{PAXOS LEARN} Message}
\label{chapter:paxos.learn.message}

The \texttt{LEARN} message triggers algorithm
\vref{algorithm:paxos.simple.learner}.

We have implemented this using multi-paxos, which will then update slots
with the number of learns.

\subsection{The \texttt{PAXOS JOIN} Message}
\label{chapter:paxos.join.message}

When the system starts up, the switches need to announce themselves to each
other and learn which ports they are on.
%
To avoid having to rely on configuration files, we built a very simple
system for announcing the presence of Paxos nodes, loosely based on the
\acf{ARP}.

Each node will send out a \texttt{JOIN} containing its own node ID and
MAC-address,, sending it out on all ports with the Ethernet broadcast
destination of \texttt{ff:ff:ff:ff:ff:ff}.

When receiving a \texttt{JOIN}, the node will store the node ID and
MAC-address in a table and pass the MAC-address and source port number ot
the L2 learning switch as well.
%
If the MAC-address is not already in the table, it will reply to the sender
with a \texttt{JOIN}.

This will continue until a node knows about at least two other nodes---the
minimum required for Paxos execution.
%
If it does not know enough nodes after some seconds, it will send out a new
\texttt{JOIN} broadcast.
%
No other Paxos messages will be processed until enough nodes are known.

Since we are only interested in Paxos phase two, we do not perform any
leader election, but it would be natural to start Paxos leader election with
prepare and promise right after the \texttt{JOIN}-phase.
%
In our setup, we have simply designated a switch as leader, and we do not
support new nodes to join the Paxos network.

\section{The \texttt{PAXOS CLIENT} Message}
\label{chapter:paxos.client.message}

The \texttt{PAXOS CLIENT} message is used for distributing client packets
among the Paxos nodes.
%
To keep consistent with the established structure, the client packet itself
starts at an offset of 64 bits from the end of the Ethernet type field.
%
The two preceding parameters are unused.

Its intended use is to forward client packets to the Paxos leader, who will
then issue an \texttt{ACCEPT} message.
%
But this means that some Paxos nodes will see the same message several
times.  Referring to figure \vref{figure:paxos.on.switches}, if switch $S_3$
receives an incoming client packet, it will forward it in a \texttt{PAXOS
CLIENT} message to $S_2$, who will forward it to the leader $S_1$.
$S_1$ will then send back a \texttt{PAXOS ACCEPT} to $S_2$, whose L2 switch
will forward it to $S_3$ again.  All containing the same client packet.

Clearly, this design could be improved.
%
One possibility would be to generate a unique identifier for each incoming
client packet.  Each \texttt{PAXOS CLIENT} message would carry it, and each
node would receive a copy of the message, storing it in a table with the
identifier as key.
%
The \texttt{PAXOS ACCEPT} message would then contain this key instead of the
full client packet.
%
The identifier could be generated on each node by using the same
technique as for $crnd$ in equation \vref{equation:crnd_mod_N}.
%
Again we must stress that---while tempting---we have decided not to spend
time on building an optimal system.
%
Our goal is to build a distributed replication system using Paxos on the
switches, and along the way we uncover important result such as these that
could be investigated further.

\section{Handling Incoming Client Packets}
\label{chapter:incoming.client}

First, when a switch gets a client packet it needs to add flow table
entries that forward it to all the other switches.
We need several OpenFlow matching rules\index{OpenFlow!matching} for all of this to work.
%
Note that all Paxos nodes except the leader will be called for
\textit{Paxos slaves} from now on.

\begin{table}[H]
  \centering
  \begin{tabular}{|l|l|}
    \hline
      \textbf{Switch} &
      \textbf{Flow Table Entry} \\

    \hline
      Leader & Store packet \\
             & Send \texttt{ACCEPT} to slaves. \\

    \hline
      Slaves & Forward to leader \\

    \hline
  \end{tabular}

  \caption{OpenFlow flow table entries.}
  \label{table:paxos.flowtable.entries}
\end{table}

Each switch need to store the full client packet and then
forward\index{forwarding} it to the other switches.

We also need entries for matching Paxos messages and their respective
actions.
%
This is done by inserting entries that match on Ethernet type
\texttt{PAXOS} and ingress port from the leader.
The action will be to go to a new entry that looks at what kind of Paxos
message we have received.

Finally, when matching on Paxos message types, we would execute 
special code using the new \texttt{run\_{}code}-action (see
    \vref{chapter:extending.openflow})
 and forward packets based on the return value from the code.

We also need new OpenFlow protocol messages\index{OpenFlow!protocol
messages} so that the controller is able to install flows with these new
actions\index{OpenFlow!extensions}.
%
However, to save time, we will simply install these flows by using the
\texttt{ovs-ofctl} command-line program from the Open vSwitch-distribution.
%
While Open vSwitch has been modified to support the new OpenFlow actions in
the switch-to-controller protocol, we would have to modify the POX framework
to be able to parse such messages, update its feature table and so on.
%
This is considered trivial to do, but time consuming and irrelevant to our
task.

\section{Paxos on the controller}

Our first step will be to implement simple Paxos \cite{Lam01} entirely in a
controller\index{controller}.

The aim is to show that a topology with a Paxos-enabled controller will
satisfy the requirements of Paxos---i.e.,~that nodes in the network reach
consensus with progress\todo{Skriv om, og sjekk at forklaring på reqs er
riktig}.

We will use Mininet\index{Mininet} \cite{Lantz:2010:NLR:1868447.1868466} to
run and simulate the \ac{SDN} and POX\index{POX} \cite{POX.1} for
implementing a Paxos in an OpenFlow controller.  POX is part of the
NOX\index{NOX}\index{controller!NOX|see{NOX}}-project \cite{Gude:2008:NTO:1384609.1384625}.  They both use
Python\index{Python}\footnote{Mininet uses Python version 2.7.}
\cite{vanRossum:2009:PRM:1610526} as the implementation language, which
means we can share some code between them.  Both projects are mature and
easy to use.  The Mininet simulation itself will run on a virtual machine
using VirtualBox\index{VirtualBox} \cite{Watson:2008:VBB:1344209.1344210}.


\section{Example of a Full Networking Flow}

In figure \vref{figure:flow.client.forwarding}, we show the complete flow of
a client packet through the system, running through the Paxos algorithm on
the switches and finally being processed and forwarded to the end-hosts.
Not shown here is any reply from these end-hosts.

\begin{figure}
  \centering
  \scriptsize
  \begin{tikzpicture}[>=stealth,x=1.2cm,y=1.2cm]
    \stdset{exec box color=white!20}
    \initstd
    \process{/S1}{$S_1$}
    \process{/S2}{$S_2$}
    \process{/S3}{$S_3$}
    \process{/c1}{$c_1$}
    \process{/hosts}{\textit{hosts}}

    % Groups
    \def\sw{/S1,/S3}
    \def\allsw{/S1,/S2,/S3}

    % Incoming client request
    \msg{/c1}{/S1}{Packet}{v}{On client}

    % ACCEPT
    \mcast{/S1}{\allsw}{Accept}{n,seq,v}{On accept, store $v$}

    % LEARN
    \alltoall{\allsw}{LEARN}{n,seq}{On learn, majority}

    % To hosts
    \mrcast{\allsw}{/hosts}{Packet}{v}{Process $v$}

    \drawtimelines
  \end{tikzpicture}
  \caption{A client $c_1$ sends a request to the system. The message is
    forwarded to the Paxos leader $S_1$, which sends the client packet as a
      parameter to an accept message.  All nodes store the packet upon
      receiving an accept.
      with wraps the message in a
      and stored on all switches.  The leader $S_1$ then sends out
      \texttt{ACCEPT} to all Paxos nodes.  The switches send \texttt{LEARN}s to
      all other switches.  When a switch has received \texttt{LEARN}s from a
      majority of nodes, it will send the message down to its
      \textit{hosts}, which then execute the client packet.  Not shown here
      is how we ensure that the client only gets back \textit{one} reply
      from the end-hosts.}
  \label{flow:simple}
\end{figure}

What we have accomplished here is using Paxos for
ordering\index{Paxos!ordering}\index{ordering} the client
requests down to the hosts, so that each host will receive packets in the
same order.  To test that the hosts have received packets in the same order,
we have run a simulation where several clients send packets to them and then
compare their output checksums using the SHA-256 algorithm.

\section{The Final Set of Flow Entries}
\label{chapter:final.flowtable}

In tables \ref{table:complete.match.leader} and
\ref{table:complete.match.slave}, we show the final table of events that the
Paxos leader and slaves will handle, respectively.
%
The tables have been implemented both on the Paxos controllers and the Paxos
switch (Open vSwitch), except for the process queue on the switch, making it
impossible to get results for round-trip times.

\begin{table}[H]
  \centering
  \begin{tabular}{|l|l|}
    \hline \textbf{Match} & \textbf{Action} \\
    \hline Packet from client & Stamp with PAXOS CLIENT Ethernet type \\
                       & Forwward to leader \\
    \hline Packet from end-host & Forward to client \\
    \hline PAXOS JOIN  & Store MAC address and node id of switch \\
                       & (\textit{Paxos-on-controller only}) \\
    \hline PAXOS CLIENT & Execute on-client \\
                        & Send PAXOS ACCEPT to all nodes with a copy of the
                        packet \\
    \hline PAXOS LEARN & Execute on-learn, forward client packet to hosts  \\
    \hline
  \end{tabular}
  \caption{The final event table for the Paxos leader.}
  \label{table:complete.match.leader}
\end{table}

In the case of the controller, these events have been implemented as
procedure calls, dispatching on Ethernet types signifying different Paxos
messages.  To forward messages on the network, OpenFlow commands were sent
down to the switch.

For the implementation on the switch (Open vSwitch), these tables have been
implemented as flow entries in the flow tables.  Everything except the
processing of the queue were imple,ented, making it impossible to measure
round-trip times.

\begin{table}[H]
  \centering
  \begin{tabular}{|l|l|}
    \hline \textbf{Match} & \textbf{Action} \\
    \hline Packet from client & Forward to leader \\
                       & Optionally, set unique packet ID \\
    \hline Packet from host & Forward to client  \\
    \hline PAXOS JOIN & Store MAC address, node id and leader-flag \\
    \hline PAXOS ACCEPT from leader & Execute program on-accept \\
                                    & Store packet \\
                                    & Send learn to all, if applicable \\
    \hline PAXOS LEARN & Execute program on-learn \\
                       & Run process queue, forwarding to hosts in-order \\
    \hline
  \end{tabular}
  \caption{The event table for the Paxos slaves.}
  \label{table:complete.match.slave}
\end{table}

Some details have been omitted from tables, and we refer to the source code
for further detail (see the appendix, chapter \ref{chapter:compiling}).


