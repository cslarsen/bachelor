\section{Using IP Fragmentation as a Buffer Mechanism}

\todo{Skriv om det her}

\begin{table}[H]
  \centering
  \begin{tabular}{|l|l|l|}
    \hline \textbf{Action} & \textbf{Parameters} & \textbf{Description} \\
    \hline Fragment packet & buffer id, fragment offset & ... \\
    \hline Defragment packet & buffer id, buffer id & ... \\
    \hline Store fragment in table & buffer id & ... \\
    \hline Retrieve fragment from table & buffer id & ... \\
    \hline
  \end{tabular}

  \caption{New OpenFlow actions.}
  \label{table:openflow.new.actions}
\end{table}

First the client sends an IP-packet to a switch.
The switch will then fragment the packet, send the first and largest
fragment to its hosts and forward it to all the other switches\todo{Dette er
litt annerledes. Og vi må sørge for at når de to andre switchene får
pakken så sender de den ikke videre}.

The end-hosts will receive an IP-fragment\index{fragmentation}, store it and
wait for the remaining fragment.
