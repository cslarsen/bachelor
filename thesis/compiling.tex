\section{Compiling}
\label{chapter:compiling}

Here follows instructions on how to compile all the code needed for running
Open vSwitch, Mininet, POX and the thesis code.

Note that the VM image should already have compiled and installed the latest
versions.  This section is only included for the sake of completeness.

\subsection{Software versions}
\label{chapter:software.versions}

\begin{table}[H]
  \centering
  \begin{tabular}{ll}
  \hline
    \textbf{Software} & \textbf{Version} \\
  \hline
    Mininet & 2.1.0+ \\
    Open vSwitch & 2.1.2 \\
    POX & 0.3.0 (dart) \\
  \hline
  \end{tabular}
  \caption{Software versions used in the thesis.}
  \label{table:software.versions}
\end{table}
\index{Mininet!version}
\index{POX!version}

The modifications we made to Open vSwitch\index{Open vSwitch!version} were
originally based on 2.0, because there were big differences in OpenFlow
upcall latencies between 2.0 and 2.1.  However, we moved these changes over
to 2.1.2, which was released after a bug was discovered based on findings by
the author\index{Open vSwitch!bug} \cite{ovs.bug}.

\subsection{Thesis code}

The thesis code consists of Mininet topologies and POX-controllers, both
written in Python, and modifications to Open vSwitch.

The Python code does not need compilation, and symlinks and paths have
already been set up correctly, provided that you use the Makefile in the
home directory of the mininet user.

\subsection{Open vSwitch}
\label{chapter:compiling.ovs}

To build Open vSwitch\index{compiling|seealso{Open vSwitch}}\index{Open
vSwitch!building} on the VM, do the following:

\begin{Verbatim}
# Remove the preinstalled ovs
$ sudo apt-get remove \
    openvswitch-common openvswitch-datapath-dkms
    openvswitch-controller openvswitch-pki openvswitch-switch

$ cd ~/ovs
$ ./boot.sh
$ ./configure --prefix=/usr \
              --with-linux=/lib/modules/$(uname -r)/build

# To compile ovs
$ make
$ make test # optional; takes some time

# To install
$ sudo make install
$ sudo make modules_install
$ sudo rmmod openvswitch
$ sudo depmod -a

# Now restart Open vSwitch
$ sudo /etc/init.d/openvswitch-controller stop
$ sudo /etc/init.d/openvswitch-switch stop

# Disable start of controller on boot
$ sudo update-rc.d openvswitch-controller disable

# And start again
$ sudo /etc/init.d/openvswitch-switch start

# Check that it's running
$ ps auxwww | grep openvswitch

# The executable ovs-controller changed name to test-controller
# in a recent ovs version, and Mininet relies on it:
$ sudo cp tests/test-controller /usr/bin/ovs-controller

# Check the version
$ sudo ovs-vsctl show
cd702e96-d4af-4803-9e2f-ecc2f7abcd6a
    ovs_version: "2.1.0"

$ modinfo openvswitch
filename: /lib/modules/3.8.0-35-generic/kernel/net/openvswitch/openvswitch.ko
license:        GPL
description:    Open vSwitch switching datapath
srcversion:     15C32AD9E04F379CAC3D68E
depends:
intree:         Y
vermagic:       3.8.0-35-generic SMP mod_unload modversions
\end{Verbatim}

The Open vSwitch-directory is a git-repository, so you can switch branches,
fetch updates and so on at will\index{git!Open vSwitch}. If you switch
branches, you need to do a full rebuild. The script
\texttt{\~{}mininet/rebuild.sh} will do this for you, but beware that it
runs \texttt{git clean -fdx} on the directory, which removes all non-tracked
files\index{git!clean}.

\subsection{POX}

POX\index{POX} is a pure Python-implementation and thus does not need any
compilation.  The directory \texttt{\~{}mininet/pox} is a
\textit{git-repository}\index{git!POX}, so you can update it at will.

\subsection{Mininet}

Mininet is mostly writte in Python but has two C-files. You can update with
\textit{git} here as well, but remember to run \texttt{sudo make install}
afterwards\index{git!Mininet}.
