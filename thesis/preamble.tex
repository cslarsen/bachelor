% Enable UTF-8 text
\usepackage[utf8]{inputenc} % enable UTF-8 text

% Adds bibliography to table of contents
\usepackage[nottoc]{tocbibind}

% Automatically collect used acronyms
\usepackage[printonlyused,withpage]{acronym}
%
% Automatically add acronyms to index, see
% http://tex.stackexchange.com/questions/22570/automatically-index-acronyms
%
% I have turned these off temporarily, because it gave pretty strange
% results.
%
%   \let\oldac\ac
%   \renewcommand*{\ac}[1]{\oldac{#1}\index{#1}}
%
% (Could also use the nomencl package here, which supports it)

% Use newlines in paragraph instead of indentation
\usepackage{parskip}

% Adds the TODO package
\usepackage[backgroundcolor=white,linecolor=red,bordercolor=red]{todonotes}

% Build index, using the new imakeidx in TeX Live, see
% http://tex.blogoverflow.com/2012/09/dont-forget-to-run-makeindex/
\usepackage{imakeidx}
\makeindex

% Uncomment this if you want to show all \index{} locations in the document
% in the margins (good for proofreading)
%\usepackage{showidx}

% Hyperlinks, clickable references
% Hidelinks turns off colored box on some PDF viewers
\usepackage[hidelinks]{hyperref}

% On Mac OSX, in ViM, when I type ALT+SHIFT+SPACE (often typed accidentally
% after writing {\some-command ...}, then LaTeX may give me the error
% "Unicode char \u8:  not set up for use with LaTeX.".  This character
% in UTF-8 actually means what the tilde (~) means in LaTeX, but *I*
% usually mean just a space.  This command replaces those with a
% normal space:
\DeclareUnicodeCharacter{00A0}{ }
% If I really want the tilde-behaviour, I can use the command
%\DeclareUnicodeCharacter{00A0}{~}
%
% The above solution was taken from:
% http://tex.stackexchange.com/questions/83440/inputenc-error-unicode-char-u8-not-set-up-for-use-with-latex

% Package for writing algorithms
\usepackage{algorithmicx}
\usepackage{algorithm}
\usepackage[noend]{algpseudocode} % noend -> no "end if" etc.

% Custom \On ... \EndOn block
\algnewcommand\algorithmicon{\textbf{on}}
\algnewcommand\algorithmicfrom{\textbf{from}}
\algblockdefx[ON]{On}{EndOn}[2]
  {\algorithmicon\ #1\ \algorithmicfrom\ #2\ \algorithmicdo}
  {\algorithmicend\ \algorithmicon}
%
% Blank end of block when using [noend] option
\makeatletter
\ifthenelse{\equal{\ALG@noend}{t}}%
  {\algtext*{EndOn}}
  {}%
\makeatother

% Custom \ForIn ... \EndForIn block
\algnewcommand\algorithmicforin{\textbf{for}}
\algnewcommand\algorithmicin{\textbf{in}}
\algblockdefx[FORIN]{ForIn}{EndForIn}[2]
  {\algorithmicforin\ #1\ \algorithmicin\ #2\ \algorithmicdo}
  {\algorithmicend\ \algorithmicforin}
%
% Blank end of block when using [noend] option
\makeatletter
\ifthenelse{\equal{\ALG@noend}{t}}%
  {\algtext*{EndForIn}}
  {}%
\makeatother

% A sendto command
\newcommand{\SendTo}[2]
  {\textbf{send}\ #1\ \textbf{to}\ #2}

\usepackage{amsmath} % for \text{}

% Better verbatim package
\usepackage{fancyvrb}

% For drawing
\usepackage{tikz}

% Above/below refs
\usepackage{varioref}

% Cells spanning vertically
\usepackage{multirow}

% For program listings
\usepackage{listings}
% Make program listings look like verbatim:
% See http://tex.stackexchange.com/questions/172702/how-can-i-make-lstlisting-look-exactly-like-verbatim?noredirect=1#172704
\lstset{
  basicstyle=\ttfamily,
  columns=fullflexible,
  keepspaces=true,
}

\usetikzlibrary{chains, scopes}

% Figures side-by-side
\usepackage{subcaption}

% Used in figures
\definecolor{verylight}{gray}{0.92}

% Hein's space time package for network flows
\usepackage{local-packages/spacetime}

% Nice units
\usepackage{SIunits}
\newcommand\ms[1]{#1~\milli\second}


% If statements ...
\usepackage{ifthen}

% Colors in tables
\usepackage{array} % to control cellspacing, see http://tex.stackexchange.com/questions/98110/white-space-between-columns-using-rowcolorsomecolor
\usepackage{xcolor,colortbl}

% Capitalize command
\makeatletter
\newcommand{\Capitalize}[1]{%
  \edef\@tempa{\expandafter\@gobble\string#1}%
  \edef\@tempb{\expandafter\@car\@tempa\@nil}%
  \edef\@tempa{\expandafter\@cdr\@tempa\@nil}%
  \uppercase\expandafter{\expandafter\def\expandafter\@tempb\expandafter{\@tempb}}%
  \@namedef{\@tempb\@tempa}{\expandafter\MakeUppercase\expandafter{#1}}}
\makeatother
