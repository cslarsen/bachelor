\chapter{Running the thesis code}
\label{chapter:install.vm}

The best way to run the thesis code is to download a virtual machine image.
We used VirtualBox\index{VirtualBox} for running this image, but it should
also work on VMWare\index{VMWare}.  It comes preloaded with
Mininet\index{Mininet}, Open vSwitch\index{Open vSwitch}, POX\index{POX},
  Wireshark\index{Wireshark} and more.

The user \texttt{mininet} has password \texttt{mininet} and has
\texttt{sudo}--rights.

\section{Downloading and verifying the VM image}

The Linux\index{Linux} VM image containing a ready--to--run version of the
code in this thesis, along with all its tools, can be downloaded from
%
\begin{center}
  \href{http://csl.name/bachelor-thesis.vm-invalid-url}{http://csl.name/bachelor-thesis.vm-invalid-url}
  \label{gpg:url}
\end{center}

To verify that this image has not been modified after the time of thesis
submission, you should download the author's GPG--key\index{GPG} (listing
\ref{gpg:key}, p.~\pageref{gpg:key}) and use it to verify the file digest
in (listing \ref{gpg:signature}, p.~\pageref{gpg:signature}).

To import the author's key, you can use \ac{GPG} or any software compatible
with PGP\index{PGP}.  Importing the key is done by running the command
\texttt{gpg --import} and pasting a copy of the author's key, ending the
input by hitting \texttt{CTRL+D}.\footnote{You can also copy the key to a
  file \texttt{key.asc} and importing it with \texttt{gpg --import key.asc}}

\begin{Verbatim}
$ gpg --import
# paste in author's key and hit CTRL+D
\end{Verbatim}

You should now see they key on your keyring.

\begin{Verbatim}
$ gpg --list-keys
\end{Verbatim}

The key's fingerprint should be \texttt{D611 0F24 4813 9908 1CFE  79BA 1AB4
  2C77 FA47 5DD2}.

\begin{lstlisting}[label={gpg:key.fingerprint}]
pub   4096R/FA475DD2 2013-04-23 [expires: 2016-04-22]
      Key fingerprint = D611 0F24 4813 9908 1CFE  79BA 1AB4 2C77 FA47 5DD2
uid                  Christian Stigen Larsen (General key) <csl@csl.name>
sub   4096R/D2495ED9 2013-04-23 [expires: 2016-04-22]
\end{lstlisting}

Finally, you need copy the VM image digest (listing \ref{gpg:signature}
\vpageref{gpg:signature}) to a file called
\texttt{bachelor-thesis.vm-invalid-url.asc}, placed in the same directory as
the downloaded VM image \texttt{bachelor-thesis.vm-invalid-url}.  You can then run
\texttt{gpg --verify bachelor-thesis.vm-invalid-url.asc} to verify the
digest against the author's key.\footnote{Note that if you haven't marked the
author's key as \textit{trusted}, you will get a warning about it.
But it should say that the signature is good.}

\begin{Verbatim}
$ gpg --verify mininet-vm-x86_64.vmdk.asc
gpg: Signature made Thu Apr 24 11:52:02 2014 CEST using RSA key ID FA475DD2
gpg: Good signature from "Christian Stigen Larsen (General key) <csl@csl.name>"
\end{Verbatim}

\begin{lstlisting}[
  float,
  caption={GPG signature for the thesis VM image.},
  label={gpg:signature}]
-----BEGIN PGP SIGNATURE-----
Version: GnuPG v1

iQIcBAABAgAGBQJTWNVMAAoJEBq0LHf6R13SBpUQAJY2LYayB8viQG3dqhMnpMWo
NXy/+4aUl4Mnr1ZmTY//d330TfX6HoM8v790cYoDQ2rjnXOBhXQ4p5RfVbEqCFfK
XuxdH+dsEXUhfh4sOj7JprQQW5EX4gKORtCm+rOwK/gTdDl5XXFckDNVdDQk6AFT
46FbJkdqfqqqpFIxxL8wiRTU4PzQFfzIwWPDKYt6RloQ65bhKuAI29St7iVqUeN9
T1Sf66IdHM0kn0nbfMlvKI9BwE2K/OWgf/Tkkf74/54Xjp47K4MreR0pqIacvK7N
tQZdG2YSfAFZUj1Gp3E+YN5v+wDN1ku6taDY0Pgwi4FSKqgmMzpaJgJs0Zi0yhGv
trAr8SdbnwV1Zl4fdUA/q48UE2RxaFWq9yXHRw3YPCirrugIOqD4foMM6ANUz82H
7+MR94RGGshpC7L7qm+rac10QvGBbmbXg1QL1AT0LH+r3D79CkGikYogss8yEbvR
QObCfC4Pnxn5RqB4+HpRAujuae9zyHWm4/hw+IeHjsPslbwNWFTz32lH/zU5wwGT
ckaFliC1IwCFBRYZ3/HXHGbSeqBXoOBg3lMTxdJ0wkYXtrDmlpkLTzNuvE/tItVh
itAEGfXpsibcTV8isyuCQQTmnqCICc41Z/iYugYqsW1+F6E1aq4G87TFKi8lUXKn
1tOAtw1HDdS9IPYsXnHQ
=GES3
-----END PGP SIGNATURE-----
\end{lstlisting}
\todo{Update the signature before submitting thesis!}

% This text is long and must be on its own page
\begin{lstlisting}[
  float,
  caption={The author's public GPG--key},
  label={gpg:key}]
-----BEGIN PGP PUBLIC KEY BLOCK-----
Version: GnuPG v1

mQINBFF21LABEACrXyM2pQ9rl0TFlubOUBb9OtUF+wxCIqFCCvgVoYNCbldXMixc
wJzv6y76m/xfrpzZjl73tAnbY3WezHpY5MtPc/OtaW0BL5nlx1i21tfoW4YpHF4N
jPXkLYFiglb43eChRE5xbH14iaJ335SLt4YKFMIArug6g9tBjyjkXvvZNXJOKxDr
jUFX13hIKzyF2x298j+sNwlXy8SUB1NM6NNrtmhD46QMqlxBn2/+U1gXQrDpBxJR
PlXomaiTNb9hEH2XQcTfPMnFbZTzGO2fWA/njzSRC3L5gwsbGronFbnkf0rN8RrF
amPOu171NsH5YhHbsVpGG6KepGA/i1zq9sPXXYHKC0NdOHdrGhEuOjDNWOy/hu0n
eldUclSmnL04QHIwGkZomrgyQT2GcNM0U5wyXZ4QHKjamNoT0etLpXPmXGMQhSVY
psVvchQfBzzohuiOqtb3KdbAqua09FB86AfmWOXsHnvv3A7q864QllpPhP0SwpZ/
bW9e/2POWYpaEZ1jWLkhj1ctnRDBP+2lvKu0zp8wvN4xRDVCPxHonbKhrWV0vCD7
UP+PAvqfhlgorXSeaSZvhMa4evSrYxZwYj52m3x3P5dSFDeFZFgrbJ/fNyRJRino
8dTYT/+ccEj1y+mGDPV2T0A60c+ZHmhpg1cshtOXk+c0qCfYEndmSAQT7QARAQAB
tDRDaHJpc3RpYW4gU3RpZ2VuIExhcnNlbiAoR2VuZXJhbCBrZXkpIDxjc2xAY3Ns
Lm5hbWU+iQI+BBMBAgAoBQJRdtSwAhsDBQkFo5qABgsJCAcDAgYVCAIJCgsEFgID
AQIeAQIXgAAKCRAatCx3+kdd0p8aD/9a6MBhfkanB4vZCKhNYjM89Oxo/E06zEAm
KCxy3N2TnRBkqwQFkflKVJe9nw2PsSITBbKDIVWV3jJ06Lgx+hVMyu5srvCPirSv
dF/0chS0tNL52ZIz13EJnprZZg80+CYPQoHPgOnS8xg+qV1bROFB5n4K+cyCy4Db
l36k316zZ042NwoaEMHqgLd7Lr55FpNyoHaGtGOLSmW4BQkfldx6G1kdJefFY43h
bj2YICuj6vY8ztmPVjrmtoiomMFAj+dIWW+z1TAsQ4lUhWpXEEIW5lNePBce2jVZ
1d+oWe9u//RzBRBKy/jI0GxE+Pq3ZdLOxM87tejYDcdb/QUQSGmQ4QugxvKXol5W
XT8wsfNm+c2amolnpVuWHMCHHZSY8PGLJwBc+oZOlEBrXxE8dU8uD3A3fBQgiABD
ETLEs0KzS9oo1m6EGrZSD0v3cbR4XtrEw9elUIhS6mwUpjoFqanNgwUXJiomsvjR
63FOw1wzO7TMz8weRr+ZUXMwvg7QEuxkIhGqnaAk5t1BMKNFneKs7NApwa2FyZ6q
oVZXG6INEee8Uw+SBph9jq9O2mBueiWVNei7+tHcZgqZAmNyPh1cwo37c4yHX707
hqzmdnnW54/HPDeIly4gC/wRQZUkWb5z981XiJgSLzppKdCyiX4ygJ730WdBjjYq
QW8ZING9xrkCDQRRdtSwARAA6xNB6yQIhYUkIZ1UfBq91iQIFrt6v0cxCUqCqFYG
y5+bnqOIoZqvNOW0oMPWKYNaCGow+lPFz8+alFBrrNzznoHYTQwNC68qaXxoIMH4
o4Ah+IK5KQ9g/iUu9fbCPcpFYH90z4TJM2uAeZO9eJQmQqaWqfj2Y9IFDNEgDEln
rq0BgpD1R+qArPsGT0i34wTQ+cO9apGZB8FH7bGqvqReBmifW9vHh7DVtA+og9Js
wWNJAaRCHv/AXnqfRxwCoeFQQGW4cncDDuaYvC6ADPPvQ9c6FzaFE36p45OKPzy/
Z1gAjDhs9dlkdtYxKj8uyMDBlDaC4J/rPfZ7kD2PFi0iXfEk0JyAXihaAjz3eJaH
xwnjRQVNrP4PW+3We3DI15VzgDmzjem2rVHBVigNhi5dFNbhSmTThQx2wRlOI57s
D3QDvJl+ppPHlP0yQETe4F9Op6diLm7jXHmvezII5RN29gKjgnz0jvXBKqHfXIco
SgB418Jp+XTzsuA/SuhuiOo/d0ac/RiM95pu8upYLbRo1VqJwQyTPqUzj0zWiSTl
waekJpwwkQONZwFtquYYVdLhR30X1GX92OfAd9JdbMGK7WR3RZfwNSUWKnCSbqoi
WAjHW9sEK7ltU8TKKXJjbseSPNKk9Khc8h87skmNofOknY03nTG50YTvHSWpxxdL
feEAEQEAAYkCJQQYAQIADwUCUXbUsAIbDAUJBaOagAAKCRAatCx3+kdd0npOEACU
Kw4AR5GBjBTcrb0JQj+YFBZO4heYA/4UMcbPqvwJzeX0nSuZ+vkB9GV08nd7/jC2
+CiHlTlrtoH40S7SdOGoZgNx7WHeHFKLo8i29lUf0ID55TOs0EctYwX/MWLWt8JJ
uZ8OXQnrXL9Mtg1yybkftMiqSmAYtWfcX6Wv86zpPzM8K4kWkd5o3NNFWBareEqL
fmdjNyyumpYX+5tHMc2v7oxG/oEC5SCVJmF5ZzFuiBBvJPPIxfYHajoR1Xz+kKf/
RDtbTJxnR0Ftfhwb5Tv4iX6rYXHwvC6bhKB4Knndba+WeN7tYKAX7rLEasdzubGj
vSZgJD3CZykk7WIq5exDHTw5E4BtgOZc5LxOf2KsWBxks6vdjbeAICA135tKxtDB
lPMytrCMNy6JNoA0fFB/KxSnOhOIxH4Ar/vXaBJxI0mQuhs9Qt+27alf6R6OgjIN
S7KQ5l3MaUFUaiRwvGJeeolT+e6X/ssLZQyWDkrNxkonZ+GhbSO5p5zgGXA3PXGg
X/18vQyTCn1jt+jzr+f/6BW+E9pqusJ4MYdDM5ThKv7TjyslMW71cdFjCHNkf1/k
9zduikQwVddgU4Ha3T6+jOY0VLncguA7UnMTOdFGdL4SchZjswDzOHPcOxC4+Un5
x7JoKcugqMAINWCfPKu5IXU6SqSkKA6ddVG/SJ5wcA==
=iAwm
-----END PGP PUBLIC KEY BLOCK-----
\end{lstlisting}
\clearpage % Clear page after this key as it's pretty long.

\section{Setting up the VM image}

\todo{Hvordan configge network, etc.}

\section{Setting up SSH}

I have added the following lines to my \texttt{~/.ssh/config}--file:

\begin{verbatim}
Host mininet
  Hostname 192.168.56.102
  User mininet
  ForwardX11 yes
  ForwardAgent yes
  RequestTTY yes
\end{verbatim}

This configuration makes sure that we forward our X11 connection and
SSH--agent to mininet.  It also makes it possible to just type \texttt{ssh
mininet} to log onto it.  You may need to change the
\texttt{Hostname}--parameter for your particular system.

You should have created an SSH--key and then upload its public part to the
mininet VM:

\begin{verbatim}
$ cat ~/.ssh/id_rsa.pub | ssh mininet cat - >> ~/.ssh/authorized_keys
\end{verbatim}

Now you should be able to log on without any password:

\begin{verbatim}
$ ssh -t mininet
\end{verbatim}

It's very important to use the \texttt{ssh \-{}t} option when specifying
remote commands to ssh. Otherwise, if you start processes by doing
\texttt{ssh mininet command} then you will leave behind dangling processes
when you hit CTRL+C (if you use the \texttt{.ssh/config} settings
above---which you are strongly recommended to do---then this should
happen automatically).

\section{Cloning the thesis code}

Then do a git clone and install the tools you need. You should already have
been given access to the private github repository by the author.

\begin{Verbatim}
$ ssh mininet
$ git clone https://github.com/cslarsen/bachelor
$ cd bachelor; ./install.sh
\end{Verbatim}

\section{Running the thesis code}

To start the first example...
\todo{Write instructions on how to run the thesis code here.}

