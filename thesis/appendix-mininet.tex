\chapter{The Thesis VM Image}
\label{chapter:install.vm}

The best way to run the thesis code is to download a virtual machine image.
We used VirtualBox\index{VirtualBox} for running this image, but it should
also work on VMWare\index{VMWare}.  It comes preloaded with
Mininet\index{Mininet}, Open vSwitch\index{Open vSwitch},
  POX\index{POX}\index{controller!POX|see{POX}},
  Wireshark\index{Wireshark} and more.

The user \texttt{mininet} has the password \texttt{mininet} and
\texttt{sudo}-rights.

\section{Setting up the Virtual Machine}

The Linux\index{Linux} VM image containing a ready-to-run version of the
code in this thesis, along with all its tools, can be downloaded from

\begin{center}
  \url{http://csl.name/thesis/mininet-vm-x86_64.vmdk}
  \label{gpg:url}
\end{center}

To verify that this image has not been modified after the time of thesis
submission, you should download the author's GPG-key\index{GPG}\index{VM
  image!verifying signature}  (listing
\ref{gpg:key}, p.~\pageref{gpg:key}) and use it to verify the file digest
in (listing \vref{gpg:signature} --- \ref{gpg:signature}, p.~\pageref{gpg:signature}).

To import the author's key, you can use \ac{GPG} or any software compatible
with \ac{PGP}\index{PGP|see{GPG}}.  Importing the key is done by running the command
\texttt{gpg --import} and pasting a copy of the author's key, ending the
input by hitting \texttt{CTRL+D}.\footnote{You can also copy the key to a
  file \texttt{key.asc} and importing it with \texttt{gpg --import key.asc}}

\begin{lstlisting}
$ gpg --import
# paste in author's key and hit CTRL+D
\end{lstlisting}
\index{GPG!importing keys}

You should now see they key on your key-ring.

\begin{lstlisting}
$ gpg --list-keys
\end{lstlisting}
\index{GPG!listing keys}

The key's fingerprint should be the same as below:

\begin{lstlisting}[label={gpg:key.fingerprint}]
pub   4096R/FA475DD2 2013-04-23 [expires: 2016-04-22]
      Key fingerprint = D611 0F24 4813 9908 1CFE  79BA 1AB4 2C77 FA47 5DD2
uid                  Christian Stigen Larsen (General key) <csl@csl.name>
sub   4096R/D2495ED9 2013-04-23 [expires: 2016-04-22]
\end{lstlisting}

Finally, you need copy the VM image digest (listing \ref{gpg:signature}
\vpageref{gpg:signature}) to a file called
\texttt{mininet-vm-x86\_{}64.vmdk.asc}, placed in the same directory as the
downloaded VM image \texttt{mininet-vm-x86\_{}64.vmdk} (from \vref{gpg.url}).
You can then run \texttt{gpg --verify mininet-vm-x86\_{}64.vmdk.asc} to verify
the digest against the author's key.\footnote{Note that if you haven't
marked the author's key as \textit{trusted}, you will get a warning about
it.  But it should say that the signature is good.}

\begin{lstlisting}[label={gpg:key.fingerprint}]
$ gpg --verify mininet-vm-x86_64.vmdk.asc
gpg: Signature made Thu Apr 24 11:52:02 2014 CEST using RSA key ID FA475DD2
gpg: Good signature from "Christian Stigen Larsen (General key) <csl@csl.name>"
\end{lstlisting}
\index{GPG!verifying signature}

\lstinputlisting[
  basicstyle=\ttfamily\footnotesize,
  caption={GPG signature for the thesis VM image.},
  label=gpg:signature]{mininet-vm-x86_64.vmdk.asc}

\lstinputlisting[
  float,
  basicstyle=\ttfamily\footnotesize,
  caption={The author's public GPG-key},
  label=gpg:key]{author-key.asc}
\clearpage

\subsection{Settings for VirtualBox}

The author's settings in \textit{VirtualBox}\index{VirtualBox|seealso{VM}}
for the Linux VM\index{VM!setting up} are given in table
\vref{table:vm.settings}.  The fields marked as \textit{needed} must be set
as shown, otherwise the VM may not work properly.

Start VirtualBox and create a new VM.  Then point to the provided VM image
(the option \textit{Use an existing virtual hard drive file})
and copy the settings in table \vref{table:vm.settings}.

When you boot the VM, you should try to ping a remote host on the internet,
then you should attempt to \texttt{ssh} into it from a terminal on the host
computer.

\begin{table}[ht]
  \centering
  \begin{tabular}{!{\vrule width -1pt}c
                  !{\vrule width -1pt}l
                  !{\vrule width -1pt}l}
  \hline
    \textbf{Needed}    & \textbf{Field}       & \textbf{Value} \\
    \hline
                       & Name                 & mininet \\
\rowcolor{verylight} * & Operating system     & Ubuntu (64 bit) \\
                       & Base memory          & 1024 MB \\
\rowcolor{verylight} * & Boot order           & Hard disk \\
                       & Acceleration         & VT-x/AMD-V, Nested Paging \\
                       & Display Video memory & 16 Mb \\
                       & IDE Secondary Master & vboxguestadditions.iso \\
                       &                      & CD/DVD \\
\rowcolor{verylight} * & SATA Port 0          & \texttt{mininet-vm-x86\_{}64.vmdk} \\
\rowcolor{verylight} * &                      & Normal, 8,00 GB \\
\rowcolor{verylight} * & Network Adapter 1    & Intel PRO/1000 MT Desktop \\
\rowcolor{verylight} * &                      & \acs{NAT} \\
\rowcolor{verylight} * &                      & MAC: \texttt{FEEDFACEBEEF} \\
\rowcolor{verylight} * & Network Adapter 2    & Intel PRO/1000 MT Desktop \\
\rowcolor{verylight} * &                      & Host-only Adapter, 'vboxnet0' \\
\rowcolor{verylight} * &                      & MAC: \texttt{0800270A8160} \\
    \hline
  \end{tabular}
  \caption{Author's settings for the VM image.}
  \label{table:vm.settings}
\end{table}
\index{VirtualBox|seealso{VM}}
\index{VM!VirtualBox settings}

It is important that you set up the network
\textit{exactly} as shown, otherwise it may not function correctly.\footnote{
  If it still does not work, make sure you have set up the guest OS
    networking settings correctly (ch.~\vref{chapter:guest.settings}).
  You may also want to edit the file
  \texttt{/etc/udev/rules.d/70-persistent-net-rules}.
  Update the corresponding MAC address and comment out all other lines, then
  reboot the VM.
   You may also need to change the VM's IP-address in
  ch.~\vref{chapter:ssh.setup}. If unsure of the IP-address, type
  \texttt{ifconfig eth1 | grep inet} to see the VM's address.
}

The \texttt{vboxguestadditions.iso} is not needed. We have used it only to
enable sharing of folders between the VM and host computer.

Remember that you can log in using the user \texttt{mininet} with the
password \texttt{mininet}.  This user should be able to get a root shell by
typing \texttt{sudo bash}.

\subsection{Network Settings}
\label{chapter:guest.settings}

To be able to use \ac{NAT} on your VM, you need to set it up on your guest
OS networking settings in VirtualBox.

In the VirtualBox manager, go to preferences, network, \textit{NAT Networks}
\index{VM!network settings}
and add a \acs{NAT}-network called \textit{NatNetworking}. Use the settings
from table \vref{table:natnetworking.settings}.

You also need to add an entry under the tab \textit{Host-only Networks}
using the settings in table
\vref{table:hostonlynetworks.settings}\index{VM!network settings}.

\begin{table}[H]
  \centering
  \begin{tabular}{ll}
    \hline \textbf{Field} & \textbf{Value} \\
    \hline
      Enable network & Yes \\
      Network name & NatNetwork \\
      Network CIDR & 10.0.2.0/24 \\
      Supports DHCP & Yes \\
    \hline
  \end{tabular}
  \caption{Settings for guest OS NAT networking.}
  \label{table:natnetworking.settings}
\end{table}
\index{VM!network settings}

\begin{table}[H]
  \centering
  \begin{tabular}{ll}
    \hline \textbf{Field} & \textbf{Value} \\
    \hline
      \textbf{Adapter} & \\
      Name & vboxnet0 \\
      IPv4 address & 192.168.56.1 \\
      IPv4 network mask & 255.255.255.0 \\
       & \\
      \textbf{DHCP server} & \\
      Enable server & Yes \\
      Server address & 192.168.56.100 \\
      Server mask & 255.255.255.0 \\
      Lower address bound & 192.168.56.101 \\
      Upper address bound & 192.168.56.254 \\
    \hline
  \end{tabular}
  \caption{Settings for guest OS Host-only Networks.}
  \label{table:hostonlynetworks.settings}
\end{table}
\index{VM!network settings}

\subsection{SSH Settings}
\label{chapter:ssh.setup}

In order to work with the VM, you need to add the following options to your
local \texttt{\~{}/.ssh/config}\index{VM!ssh configuration}\index{\texttt{.ssh/config}|see{VM}}

\begin{verbatim}
Host mininet
  Hostname 192.168.56.102
  User mininet
  ForwardX11 yes
  ForwardAgent yes
  RequestTTY yes
\end{verbatim}
\index{VM!ssh}

X11-forwarding\index{VM!X11 forwarding}\index{X11} is required in case you
want to start xterms on Mininet
nodes or run Wireshark.\index{VM!Wireshark}  Note that it also forwards your
ssh-agent---you may not strictly need this.  The \texttt{RequestTTY}-option
is \textbf{very important}, because it lets us start terminal programs on the remote
host. Without it, some of the examples here will leave processes running in
the background on the remote host when you SIGINT them (CTRL+D).\footnote{If
you don't use this option, you can manually type \texttt{ssh -t} to request
TTYs correctly.}

You may need to change the \texttt{Hostname}-parameter for your particular
system.  Doing this correctly saves you time when the host computer changes
networks.

To be able to log on password-less, you need to upload your public key:

\begin{verbatim}
$ cat ~/.ssh/id_rsa.pub | ssh mininet "cat - >> ~/.ssh/authorized_keys"
\end{verbatim}
\index{VM!ssh}

Boot the VM and make sure you can log on to it without typing a password:

\begin{verbatim}
$ ssh mininet
\end{verbatim}

\section{Compiling}
\label{chapter:compiling}

Here follows instructions on how to compile all the code needed for running
Open vSwitch, Mininet, POX and the thesis code.

Note that the VM image should already have compiled and installed the latest
versions.  This section is only included for the sake of completeness.

\subsection{Software versions}
\label{chapter:software.versions}

\begin{table}[H]
  \centering
  \begin{tabular}{ll}
  \hline
    \textbf{Software} & \textbf{Version} \\
  \hline
    Mininet & 2.1.0+ \\
    Open vSwitch & 2.1.2 \\
    POX & 0.3.0 (dart) \\
  \hline
  \end{tabular}
  \caption{Software versions used in the thesis.}
  \label{table:software.versions}
\end{table}
\index{Mininet!version}
\index{POX!version}

The modifications we made to Open vSwitch\index{Open vSwitch!version} were
originally based on 2.0, because there were big differences in OpenFlow
upcall latencies between 2.0 and 2.1.  However, we moved these changes over
to 2.1.2, which was released after a bug was discovered based on findings by
the author\index{Open vSwitch!bug} \cite{ovs.bug}.

\subsection{Thesis code}

The thesis code consists of Mininet topologies and POX-controllers, both
written in Python, and modifications to Open vSwitch.

The Python code does not need compilation, and symlinks and paths have
already been set up correctly, provided that you use the Makefile in the
home directory of the mininet user.

\subsection{Open vSwitch}
\label{chapter:compiling.ovs}

To build Open vSwitch\index{compiling|seealso{Open vSwitch}}\index{Open
vSwitch!building} on the VM, do the following:

\begin{Verbatim}
# Remove the preinstalled ovs
$ sudo apt-get remove \
    openvswitch-common openvswitch-datapath-dkms
    openvswitch-controller openvswitch-pki openvswitch-switch

$ cd ~/ovs
$ ./boot.sh
$ ./configure --prefix=/usr \
              --with-linux=/lib/modules/$(uname -r)/build

# To compile ovs
$ make
$ make test # optional; takes some time

# To install
$ sudo make install
$ sudo make modules_install
$ sudo rmmod openvswitch
$ sudo depmod -a

# Now restart Open vSwitch
$ sudo /etc/init.d/openvswitch-controller stop
$ sudo /etc/init.d/openvswitch-switch stop

# Disable start of controller on boot
$ sudo update-rc.d openvswitch-controller disable

# And start again
$ sudo /etc/init.d/openvswitch-switch start

# Check that it's running
$ ps auxwww | grep openvswitch

# The executable ovs-controller changed name to test-controller
# in a recent ovs version, and Mininet relies on it:
$ sudo cp tests/test-controller /usr/bin/ovs-controller

# Check the version
$ sudo ovs-vsctl show
cd702e96-d4af-4803-9e2f-ecc2f7abcd6a
    ovs_version: "2.1.0"

$ modinfo openvswitch
filename: /lib/modules/3.8.0-35-generic/kernel/net/openvswitch/openvswitch.ko
license:        GPL
description:    Open vSwitch switching datapath
srcversion:     15C32AD9E04F379CAC3D68E
depends:
intree:         Y
vermagic:       3.8.0-35-generic SMP mod_unload modversions
\end{Verbatim}

The Open vSwitch-directory is a git-repository, so you can switch branches,
fetch updates and so on at will\index{git!Open vSwitch}. If you switch
branches, you need to do a full rebuild. The script
\texttt{\~{}mininet/rebuild.sh} will do this for you, but beware that it
runs \texttt{git clean -fdx} on the directory, which removes all non-tracked
files\index{git!clean}.

\subsection{POX}

POX\index{POX} is a pure Python-implementation and thus does not need any
compilation.  The directory \texttt{\~{}mininet/pox} is a
\textit{git-repository}\index{git!POX}, so you can update it at will.

\subsection{Mininet}

Mininet is mostly written in Python but has two C-files. You can update with
\textit{git} here as well, but remember to run \texttt{sudo make install}
afterwards\index{git!Mininet}.


\section{Running the Code}
\label{chapter:appendix.benchmark}

After setting up the VM image correctly (chapters \ref{chapter:install.vm}
and \ref{chapter:ssh.setup}), you may want to restart your Mininet
before running benchmarks.  This is to make sure that no processes from
previous runs are hanging in the background.\footnote{There may even be
hanging processes from the time the VM image was uploaded.} The test code
mostly takes care of this, but to be on the safe side, reboot with

\begin{Verbatim}
$ ssh mininet sudo shutdown -r now
\end{Verbatim}

\subsection{Baseline benchmarks}
\label{chapter:appendix.baseline.benchmark}

To run this benchmark\index{benchmark}, run the following on your local computer

\begin{Verbatim}
$ ssh mininet make bench-baseline
$ ssh mininet make bench-baseline-noflows
\end{Verbatim}

If you prefer to run Mininet and the controller in separate terminals, you
can start each one independently:

\begin{Verbatim}
# Terminal 1 (start first)
$ ssh mininet make bench-baseline-pox

# Terminal 2 (start after POX is up)
$ ssh mininet make bench-baseline-mininet
\end{Verbatim}

After the test runs complete, there are two result files that you can
download locally:

\begin{Verbatim}
$ scp mininet:~/pings.txt mininet:~/pings-noflows.txt .
\end{Verbatim}

\subsection{Running Paxos}
\label{chapter:running.paxos}

Log on to the VM using two different terminals.  Start Mininet in one
window with \texttt{make paxos-net} and the controllers in another
with \texttt{make paxos-pox-noflows}.  This starts the controllers in a mode
where they will not add flows.

You should have X11\index{X11}\index{VM!X11 forwarding} set up as well (see
chapter \ref{chapter:ssh.setup}).  Open up two X11 terminals on

The network should initialize and the Paxos controllers should announce
themselves to each other.  You can now run
Mininet-commands\index{Mininet!commands} such as: \texttt{nodes} to see all
nodes on the network, \texttt{net} to see their links and \texttt{pingall}
\index{Mininet!pingall} to have all nodes ping each other.\footnote{The 
Mininet boot-script should do this automatically, as it will make sure that
the controllers learn which ports different Ethernet addresses can be
reached on.}

To run a command on a node, just type the node's name along with an ordinary
shell command.  For instance,

\begin{Verbatim}
paxos/mininet> h1 ping c9
\end{Verbatim}

To test things, we can start a web-server\index{Mininet!web-server} on
\texttt{h9}. The best way to do this is in \texttt{h9}'s X11-terminal:

\begin{Verbatim}
root@mininet-vm:~# python -m SimpleHTTPServer
Serving HTTP on 0.0.0.0 port 8000 ...
10.0.0.1 - - [06/May/2014 19:09:48] "GET / HTTP/1.1" 200 -
10.0.0.1 - - [06/May/2014 19:09:55] "GET / HTTP/1.1" 200 -
10.0.0.1 - - [06/May/2014 19:09:58] "GET / HTTP/1.1" 200 -
\end{Verbatim}

This small web-server will list all files in its current directory, and make
them available for download.
On \texttt{c1}'s X11-terminal, you can use \texttt{curl}\index{curl} to fetch
web-pages from \texttt{h9}.\footnote{You can also do this from Mininet's
command-line prompt by typing \texttt{c1 curl http://10.0.0.12:8000}.}

\begin{Verbatim}
root@mininet-vm:~# curl http://10.0.0.12:8000
\end{Verbatim}

You should get some HTML\index{HTML}-output.  In the controller console, notice
all the log messages.  What is happening is that each individual
\acs{TCP}-packet\index{TCP} will be ordered by the Paxos-system.

To get a rough indication of \acf{RTT}, you can run the \texttt{time}\index{\texttt{time}}-command:

\begin{Verbatim}
$ /usr/bin/time -p curl http://10.0.0.12:8000
\end{Verbatim}

\subsection{Monitoring Network Traffic}
\label{chapter:tcpdump}

If you want to monitor network traffic, you can use the
\texttt{tcpdump}\index{tcpdump}\index{monitoring network traffic}
command.  You specify an interface to listen to with the
\texttt{-i} option (e.g., \textit{S1-eth1};
\texttt{ifconfig}\index{ifconfig} gives a full list) and you can
optionally give packet filtering rules using the
\acf{BPF}-syntax\index{Berkeley Packet Filter}\index{BPF|see{Berkeley
Packet Filter}} \cite{McCanne:1993:BPF:1267303.1267305}.  For instance,

\begin{Verbatim}
$ sudo tcpdump -nNeXS -s64 -iany \
    "(port not 22) and (port not 6633) and (dst 10.0.0.1)"
\end{Verbatim}

specifies in \acs{BPF} to capture packets bound for \texttt{10.0.0.1},
except if the source or destination port is 22
(\textit{ssh}\index{tcpdump!filtering}) or 6633\index{controller!traffic
monitoring} (the default
controller port).  The option \texttt{-iany} instructs \texttt{tcpdump} to
capture on all interfaces.  The remaining options are explained in the
manual for \texttt{tcpdump}.\footnote{\texttt{man 1 tcpdump}}
Example output of the above command is given below.

\begin{Verbatim}[fontsize=\footnotesize]
tcpdump: verbose output suppressed, use -v or -vv for full protocol decode
listening on any, link-type LINUX_SLL (Linux cooked), capture size 64 bytes
20:34:00.063947   P 0e:df:8b:76:a5:1a ethertype IPv4 (0x0800), 
  length 100: 10.0.0.9 > 10.0.0.1: ICMP echo reply, id 15566, seq 868, length 64
  0x0000:  4500 0054 54e4 0000 4001 11bc 0a00 0009  E..TT...@.......
  0x0010:  0a00 0001 0000 0390 3cce 0364 9893 6253  ........<..d..bS
  0x0020:  0000 0000 0284 0000 0000 0000 1011 1213  ................
20:34:00.070384 Out 0e:df:8b:76:a5:1a ethertype IPv4 (0x0800), 
  length 100: 10.0.0.9 > 10.0.0.1: ICMP echo reply, id 15566, seq 868, length 64
  0x0000:  4500 0054 54e4 0000 4001 11bc 0a00 0009  E..TT...@.......
  0x0010:  0a00 0001 0000 0390 3cce 0364 9893 6253  ........<..d..bS
  0x0020:  0000 0000 0284 0000 0000 0000 1011 1213  ................
\end{Verbatim}

Our boot-scripts for Mininet usually start of with the
\texttt{pingall}-command, which makes every node ping all other nodes.
If you want to capture them, you need to start up Mininet first, then
tcpdump and then the controller last.

Mininet will create network interfaces for each link in its topology.
So, after starting Mininet, you can see the interfaces for $S_1$ by typing

\begin{Verbatim}
$ ifconfig | grep S1
S1        Link encap:Ethernet  HWaddr 3a:6a:80:cd:b1:43
S1-eth1   Link encap:Ethernet  HWaddr 7e:56:63:c8:4a:f8
S1-eth2   Link encap:Ethernet  HWaddr 96:66:dd:b5:1c:c7
S1-eth3   Link encap:Ethernet  HWaddr 8e:9f:d0:8c:10:71
S1-eth4   Link encap:Ethernet  HWaddr 86:30:25:58:12:7e
\end{Verbatim}

The interfaces starting with \texttt{S1-eth}\dots~are the port interfaces.
To monitor one of them, simply type

\begin{Verbatim}[fontsize=\footnotesize]
$ sudo tcpdump -nev -iS1-eth1
tcpdump: WARNING: S1-eth1: no IPv4 address assigned
tcpdump: listening on S1-eth1, link-type EN10MB (Ethernet), capture size 65535 bytes
11:50:53.967333 06:b6:0d:37:d0:cd > ff:ff:ff:ff:ff:ff, ethertype ARP (0x0806), length 42:
    Ethernet (len 6), IPv4 (len 4), Request who-has 10.0.0.9 tell 10.0.0.1, length 28
[...]
11:51:14.636681 9e:df:10:0b:a7:da > 06:b6:0d:37:d0:cd, ethertype IPv4 (0x0800), length 98:
    (tos 0x0, ttl 64, id 32782, offset 0, flags [DF], proto ICMP (1), length 84)
    10.0.0.9 > 10.0.0.1: ICMP echo request, id 13921, seq 1, length 64
11:51:14.641836 06:b6:0d:37:d0:cd > 9e:df:10:0b:a7:da, ethertype IPv4 (0x0800), length 98:
    (tos 0x0, ttl 64, id 36379, offset 0, flags [none], proto ICMP (1), length 84)
    10.0.0.1 > 10.0.0.9: ICMP echo reply, id 13921, seq 1, length 64
\end{Verbatim}
