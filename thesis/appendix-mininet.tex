\chapter{Running the thesis code}
\label{chapter:install.vm}

The best way to run the thesis code is to download a virtual machine image.
We used VirtualBox\index{VirtualBox} for running this image, but it should
also work on VMWare\index{VMWare}.  It comes preloaded with
Mininet\index{Mininet}, Open vSwitch\index{Open vSwitch},
  POX\index{POX}\index{controller|seealso{POX}},
  Wireshark\index{Wireshark} and more.

The user \texttt{mininet} has password \texttt{mininet} and has
\texttt{sudo}-rights.

\section{Downloading and verifying the VM image}

The Linux\index{Linux} VM image containing a ready-to-run version of the
code in this thesis, along with all its tools, can be downloaded from

\begin{center}
  \url{http://csl.name/thesis/mininet-vm-x86_64.vmdk}
  \label{gpg:url}
\end{center}

To verify that this image has not been modified after the time of thesis
submission, you should download the author's GPG-key\index{GPG} (listing
\ref{gpg:key}, p.~\pageref{gpg:key}) and use it to verify the file digest
in (listing \ref{gpg:signature}, p.~\pageref{gpg:signature}).

To import the author's key, you can use \ac{GPG} or any software compatible
with PGP\index{PGP}.  Importing the key is done by running the command
\texttt{gpg --import} and pasting a copy of the author's key, ending the
input by hitting \texttt{CTRL+D}.\footnote{You can also copy the key to a
  file \texttt{key.asc} and importing it with \texttt{gpg --import key.asc}}
%
\begin{Verbatim}
$ gpg --import
# paste in author's key and hit CTRL+D
\end{Verbatim}
%
You should now see they key on your key-ring.
%
\begin{Verbatim}
$ gpg --list-keys
\end{Verbatim}
%
The key's fingerprint should be the same as below:
%
\begin{lstlisting}[label={gpg:key.fingerprint}]
pub   4096R/FA475DD2 2013-04-23 [expires: 2016-04-22]
      Key fingerprint = D611 0F24 4813 9908 1CFE  79BA 1AB4 2C77 FA47 5DD2
uid                  Christian Stigen Larsen (General key) <csl@csl.name>
sub   4096R/D2495ED9 2013-04-23 [expires: 2016-04-22]
\end{lstlisting}

Finally, you need copy the VM image digest (listing \ref{gpg:signature}
\vpageref{gpg:signature}) to a file called
\texttt{mininet-vm-x86\_{}64.vmdk.asc}, placed in the same directory as the
downloaded VM image \texttt{mininet-vm-x86\_{}64.vmdk} (from \vref{gpg.url}).
You can then run \texttt{gpg --verify mininet-vm-x86\_{}64.vmdk.asc} to verify
the digest against the author's key.\footnote{Note that if you haven't
marked the author's key as \textit{trusted}, you will get a warning about
it.  But it should say that the signature is good.}

\begin{Verbatim}
$ gpg --verify mininet-vm-x86_64.vmdk.asc
gpg: Signature made Thu Apr 24 11:52:02 2014 CEST using RSA key ID FA475DD2
gpg: Good signature from "Christian Stigen Larsen (General key) <csl@csl.name>"
\end{Verbatim}

\lstinputlisting[
  caption={GPG signature for the thesis VM image.},
  label={gpg:signature}]{mininet-vm-x86_64.vmdk.asc}

% This text is long and must be on its own page
\begin{lstlisting}[
  float,
  caption={The author's public GPG-key},
  label={gpg:key}]
-----BEGIN PGP PUBLIC KEY BLOCK-----
Version: GnuPG v1

mQINBFF21LABEACrXyM2pQ9rl0TFlubOUBb9OtUF+wxCIqFCCvgVoYNCbldXMixc
wJzv6y76m/xfrpzZjl73tAnbY3WezHpY5MtPc/OtaW0BL5nlx1i21tfoW4YpHF4N
jPXkLYFiglb43eChRE5xbH14iaJ335SLt4YKFMIArug6g9tBjyjkXvvZNXJOKxDr
jUFX13hIKzyF2x298j+sNwlXy8SUB1NM6NNrtmhD46QMqlxBn2/+U1gXQrDpBxJR
PlXomaiTNb9hEH2XQcTfPMnFbZTzGO2fWA/njzSRC3L5gwsbGronFbnkf0rN8RrF
amPOu171NsH5YhHbsVpGG6KepGA/i1zq9sPXXYHKC0NdOHdrGhEuOjDNWOy/hu0n
eldUclSmnL04QHIwGkZomrgyQT2GcNM0U5wyXZ4QHKjamNoT0etLpXPmXGMQhSVY
psVvchQfBzzohuiOqtb3KdbAqua09FB86AfmWOXsHnvv3A7q864QllpPhP0SwpZ/
bW9e/2POWYpaEZ1jWLkhj1ctnRDBP+2lvKu0zp8wvN4xRDVCPxHonbKhrWV0vCD7
UP+PAvqfhlgorXSeaSZvhMa4evSrYxZwYj52m3x3P5dSFDeFZFgrbJ/fNyRJRino
8dTYT/+ccEj1y+mGDPV2T0A60c+ZHmhpg1cshtOXk+c0qCfYEndmSAQT7QARAQAB
tDRDaHJpc3RpYW4gU3RpZ2VuIExhcnNlbiAoR2VuZXJhbCBrZXkpIDxjc2xAY3Ns
Lm5hbWU+iQI+BBMBAgAoBQJRdtSwAhsDBQkFo5qABgsJCAcDAgYVCAIJCgsEFgID
AQIeAQIXgAAKCRAatCx3+kdd0p8aD/9a6MBhfkanB4vZCKhNYjM89Oxo/E06zEAm
KCxy3N2TnRBkqwQFkflKVJe9nw2PsSITBbKDIVWV3jJ06Lgx+hVMyu5srvCPirSv
dF/0chS0tNL52ZIz13EJnprZZg80+CYPQoHPgOnS8xg+qV1bROFB5n4K+cyCy4Db
l36k316zZ042NwoaEMHqgLd7Lr55FpNyoHaGtGOLSmW4BQkfldx6G1kdJefFY43h
bj2YICuj6vY8ztmPVjrmtoiomMFAj+dIWW+z1TAsQ4lUhWpXEEIW5lNePBce2jVZ
1d+oWe9u//RzBRBKy/jI0GxE+Pq3ZdLOxM87tejYDcdb/QUQSGmQ4QugxvKXol5W
XT8wsfNm+c2amolnpVuWHMCHHZSY8PGLJwBc+oZOlEBrXxE8dU8uD3A3fBQgiABD
ETLEs0KzS9oo1m6EGrZSD0v3cbR4XtrEw9elUIhS6mwUpjoFqanNgwUXJiomsvjR
63FOw1wzO7TMz8weRr+ZUXMwvg7QEuxkIhGqnaAk5t1BMKNFneKs7NApwa2FyZ6q
oVZXG6INEee8Uw+SBph9jq9O2mBueiWVNei7+tHcZgqZAmNyPh1cwo37c4yHX707
hqzmdnnW54/HPDeIly4gC/wRQZUkWb5z981XiJgSLzppKdCyiX4ygJ730WdBjjYq
QW8ZING9xrkCDQRRdtSwARAA6xNB6yQIhYUkIZ1UfBq91iQIFrt6v0cxCUqCqFYG
y5+bnqOIoZqvNOW0oMPWKYNaCGow+lPFz8+alFBrrNzznoHYTQwNC68qaXxoIMH4
o4Ah+IK5KQ9g/iUu9fbCPcpFYH90z4TJM2uAeZO9eJQmQqaWqfj2Y9IFDNEgDEln
rq0BgpD1R+qArPsGT0i34wTQ+cO9apGZB8FH7bGqvqReBmifW9vHh7DVtA+og9Js
wWNJAaRCHv/AXnqfRxwCoeFQQGW4cncDDuaYvC6ADPPvQ9c6FzaFE36p45OKPzy/
Z1gAjDhs9dlkdtYxKj8uyMDBlDaC4J/rPfZ7kD2PFi0iXfEk0JyAXihaAjz3eJaH
xwnjRQVNrP4PW+3We3DI15VzgDmzjem2rVHBVigNhi5dFNbhSmTThQx2wRlOI57s
D3QDvJl+ppPHlP0yQETe4F9Op6diLm7jXHmvezII5RN29gKjgnz0jvXBKqHfXIco
SgB418Jp+XTzsuA/SuhuiOo/d0ac/RiM95pu8upYLbRo1VqJwQyTPqUzj0zWiSTl
waekJpwwkQONZwFtquYYVdLhR30X1GX92OfAd9JdbMGK7WR3RZfwNSUWKnCSbqoi
WAjHW9sEK7ltU8TKKXJjbseSPNKk9Khc8h87skmNofOknY03nTG50YTvHSWpxxdL
feEAEQEAAYkCJQQYAQIADwUCUXbUsAIbDAUJBaOagAAKCRAatCx3+kdd0npOEACU
Kw4AR5GBjBTcrb0JQj+YFBZO4heYA/4UMcbPqvwJzeX0nSuZ+vkB9GV08nd7/jC2
+CiHlTlrtoH40S7SdOGoZgNx7WHeHFKLo8i29lUf0ID55TOs0EctYwX/MWLWt8JJ
uZ8OXQnrXL9Mtg1yybkftMiqSmAYtWfcX6Wv86zpPzM8K4kWkd5o3NNFWBareEqL
fmdjNyyumpYX+5tHMc2v7oxG/oEC5SCVJmF5ZzFuiBBvJPPIxfYHajoR1Xz+kKf/
RDtbTJxnR0Ftfhwb5Tv4iX6rYXHwvC6bhKB4Knndba+WeN7tYKAX7rLEasdzubGj
vSZgJD3CZykk7WIq5exDHTw5E4BtgOZc5LxOf2KsWBxks6vdjbeAICA135tKxtDB
lPMytrCMNy6JNoA0fFB/KxSnOhOIxH4Ar/vXaBJxI0mQuhs9Qt+27alf6R6OgjIN
S7KQ5l3MaUFUaiRwvGJeeolT+e6X/ssLZQyWDkrNxkonZ+GhbSO5p5zgGXA3PXGg
X/18vQyTCn1jt+jzr+f/6BW+E9pqusJ4MYdDM5ThKv7TjyslMW71cdFjCHNkf1/k
9zduikQwVddgU4Ha3T6+jOY0VLncguA7UnMTOdFGdL4SchZjswDzOHPcOxC4+Un5
x7JoKcugqMAINWCfPKu5IXU6SqSkKA6ddVG/SJ5wcA==
=iAwm
-----END PGP PUBLIC KEY BLOCK-----
\end{lstlisting}
\clearpage % Clear page after this key as it's pretty long.

\section{Setting up the VM image}

The author's settings in \textit{VirtualBox}\index{VirtualBox|seealso{VM}}
for the Linux VM\index{VM!setting up} are given in table
\vref{table:vm.settings}.  The fields marked as \textit{needed} must be set
as shown, otherwise the VM may not work properly.

Start VirtualBox and create a new VM.  Then point to the provided VM image
(the option \textit{Use an existing virtual hard drive file})
and copy the settings in table \vref{table:vm.settings}.

When you boot the VM, you should try to ping a remote host on the internet,
then you should attempt to \texttt{ssh} into it from a terminal on the host
computer.

% TODO: Fix the VERTICAL spacing as well, it is still white
\begin{table}[ht]
  \begin{tabular}{!{\vrule width -1pt}c
                  !{\vrule width -1pt}l
                  !{\vrule width -1pt}l}
  \hline
    \textbf{Needed}    & \textbf{Field}       & \textbf{Value} \\
    \hline
                       & Name                 & mininet \\
\rowcolor{verylight} * & Operating system     & Ubuntu (64 bit) \\
                       & Base memory          & 1024 MB \\
\rowcolor{verylight} * & Boot order           & Hard disk \\
                       & Acceleration         & VT-x/AMD-V, Nested Paging \\
                       & Display Video memory & 16 Mb \\
                       & IDE Secondary Master & vboxguestadditions.iso \\
                       &                      & CD/DVD \\
\rowcolor{verylight} * & SATA Port 0          & \texttt{mininet-vm-x86\_{}64.vmdk} \\
\rowcolor{verylight} * &                      & Normal, 8,00 GB \\
\rowcolor{verylight} * & Network Adapter 1    & Intel PRO/1000 MT Desktop \\
\rowcolor{verylight} * &                      & NAT \\
\rowcolor{verylight} * &                      & MAC: \texttt{FEEDFACEBEEF} \\
\rowcolor{verylight} * & Network Adapter 2    & Intel PRO/1000 MT Desktop \\
\rowcolor{verylight} * &                      & Host-only Adapter, 'vboxnet0' \\
\rowcolor{verylight} * &                      & MAC: \texttt{0800270A8160} \\
    \hline
  \end{tabular}
  \caption{Author's settings for the VM image.}
  \label{table:vm.settings}
\end{table}
\index{VirtualBox|seealso{VM}}
\index{VM!VirtualBox settings}

It is important that you set up the network
\textit{exactly} as shown, otherwise it may not function correctly.\footnote{
  If it stil does not work, make sure you have set up the guest OS
    networking settings correctly (ch.~\vref{chapter:guest.settings}).
  You may also want to edit the file
  \texttt{/etc/udev/rules.d/70-persistent-net-rules}.
  Update the corresponding MAC address and comment out all other lines, then
  reboot the VM.
   You may also need to change the VM's IP-address in
  ch.~\vref{chapter:ssh.setup}. If unsure of the IP-address, type
  \texttt{ifconfig eth1 | grep inet} to see the VM's address.
}

The \texttt{vboxguestadditions.iso} is not needed. We have used it only to
enable sharing of folders between the VM and host computer.

Remember that you can log in using the user \texttt{mininet} with the
password \texttt{mininet}.  This user should be able to get a root shell by
typing \texttt{sudo bash}.

\subsection{Guest OS network settings}
\label{chapter:guest.settings}

To be able to use \acs{NAT} on your VM, you need to set it up on your guest
OS networking settings in VirtualBox.

In the VirtualBox manager, go to preferences, network, \textit{NAT Networks}
and add a \acs{NAT}-network called \textit{NatNetworking}. Use the settings
from table \vref{table:natnetworking.settings}.

You also need to add an entry under the tab \textit{Host-only Networks}
using the settings in table \vref{table:hostonlynetworks.settings}.

\begin{table}[H]
  \centering
  \begin{tabular}{ll}
    \hline \textbf{Field} & \textbf{Value} \\
    \hline
      Enable network & Yes \\
      Network name & NatNetwork \\
      Network CIDR & 10.0.2.0/24 \\
      Supports DHCP & Yes \\
    \hline
  \end{tabular}
  \caption{Settings for guest OS NAT networking.}
  \label{table:natnetworking.settings}
\end{table}
\index{VM!network settings}

\begin{table}[H]
  \centering
  \begin{tabular}{ll}
    \hline \textbf{Field} & \textbf{Value} \\
    \hline
      \textbf{Adapter} & \\
      Name & vboxnet0 \\
      IPv4 address & 192.168.56.1 \\
      IPv4 network mask & 255.255.255.0 \\
       & \\
      \textbf{DHCP server} & \\
      Enable server & Yes \\
      Server address & 192.168.56.100 \\
      Server mask & 255.255.255.0 \\
      Lower address bound & 192.168.56.101 \\
      Upper address bound & 192.168.56.254 \\
    \hline
  \end{tabular}
  \caption{Settings for guest OS Host-only Networks.}
  \label{table:hostonlynetworks.settings}
\end{table}
\index{VM!network settings}

\subsection{Setting up SSH}
\label{chapter:ssh.setup}

In order to work with the VM, you need to add the following optios to your
local \texttt{\~{}/.ssh/config}.

\begin{verbatim}
Host mininet
  Hostname 192.168.56.102
  User mininet
  ForwardX11 yes
  ForwardAgent yes
  RequestTTY yes
\end{verbatim}
\index{VM!ssh}

X11-forwarding\index{VM!X11 forwarding} is required in case you want to start xterms on Mininet
nodes or run Wireshark.\index{VM!Wireshark}  Note that it also forwards your
ssh-agent---you may not strictly need this.  The \texttt{RequestTTY}-option
is \textbf{very important}, because it lets us start terminal programs on the remote
host. Without it, some of the examples here will leave processes running in
the background on the remote host when you SIGINT them (CTRL+D).\footnote{If
you don't use this option, you can manually type \texttt{ssh -t} to request
TTYs correctly.}

You may need to change the \texttt{Hostname}-parameter for your particular
system.  Doing this correctly saves you time when the host computer changes
networks.

To be able to logon passwordless, you need to upload your public key:

\begin{verbatim}
$ cat ~/.ssh/id_rsa.pub | ssh mininet "cat - >> ~/.ssh/authorized_keys"
\end{verbatim}
\index{VM!ssh}

Boot the VM and make sure you can log on to it without typing a password:

\begin{verbatim}
$ ssh mininet
\end{verbatim}

\section{The thesis source code}

You can find the thesis source code on the Mininet VM under
\texttt{\~{}/bach}.

\section{Running the thesis code}

To start the first example...
\todo{Write instructions on how to run the thesis code here.}

\section{Building code}
\label{chapter:compiling}

\subsection{Open vSwitch}
\label{chapter:compiling.ovs}

To build Open vSwitch\index{compiling|seealso{Open vSwitch}}\index{Open
vSwitch!building} on the VM, do the following:

\begin{Verbatim}
# Remove the preinstalled ovs
$ sudo apt-get remove \
    openvswitch-common openvswitch-datapath-dkms
    openvswitch-controller openvswitch-pki openvswitch-switch

$ cd ~/ovs
$ ./boot.sh
$ ./configure --prefix=/usr \
              --with-linux=/lib/modules/$(uname -r)/build

# To compile ovs
$ make
$ make test # optional; takes some time

# To install
$ sudo make install
$ sudo make modules_install
$ sudo rmmod openvswitch
$ sudo depmod -a

# Now restart Open vSwitch
$ sudo /etc/init.d/openvswitch-controller stop
$ sudo /etc/init.d/openvswitch-switch stop

# Disable start of controller on boot
$ sudo update-rc.d openvswitch-controller disable

# And start again
$ sudo /etc/init.d/openvswitch-switch start

# Check that it's running
$ ps auxwww | grep openvswitch

# The executable ovs-controller changed name to test-controller
# in a recent ovs version, and Mininet relies on it:
$ sudo cp tests/test-controller /usr/bin/ovs-controller

# Check the version
$ sudo ovs-vsctl show
cd702e96-d4af-4803-9e2f-ecc2f7abcd6a
    ovs_version: "2.1.0"

$ modinfo openvswitch
filename: /lib/modules/3.8.0-35-generic/kernel/net/openvswitch/openvswitch.ko
license:        GPL
description:    Open vSwitch switching datapath
srcversion:     15C32AD9E04F379CAC3D68E
depends:
intree:         Y
vermagic:       3.8.0-35-generic SMP mod_unload modversions
\end{Verbatim}

\subsection{POX}

\todo{hvordan kompilere pox}

\subsection{Thesis code}

The thesis code is mostly written in Python, so no steps are necessary.
The only thing to remember is to use the correct PYTHONPATH.  You are
recommended to use the makefiles for this. We have also added symlinks where
needed, so you should not need to worry about it.

\todo{Hvordan bygge koden vår, C-koden, osv.}

\section{Running the benchmarks}
\label{chapter:appendix.benchmark}

To run the benchmarks yourself, you first need to set up the Linux VM image
by following the instructions in \ref{chapter:install.vm}
\vpageref{chapter:install.vm}.

You absolutely need to have set up \texttt{ssh} as shown in
\ref{chapter:ssh.setup} \vpageref{chapter:ssh.setup}, because we need
correct allocations of TTYs.

When you have done that, you may want to restart your Mininet before running
benchmarks.  This is to make sure that no processes from previous runs are
hanging in the background.\footnote{There may even be hanging processes from
the time the VM image was uploaded.} The test code mostly takes care of
this, but to be on the safe side, reboot with

\begin{Verbatim}
$ ssh mininet
$ sudo shutdown -r now
\end{Verbatim}

\subsection{Baseline benchmark}
\label{chapter:appendix.baseline.benchmark}

To run this benchmark\index{benchmark}, run the following on your local computer

\begin{Verbatim}
$ ssh mininet make bench-baseline
$ ssh mininet make bench-baseline-noflows
\end{Verbatim}

If you prefer to run Mininet and the controller in separate terminals, you
can start each one independently:

\begin{Verbatim}
# Terminal 1 (start first)
$ ssh mininet make bench-baseline-pox

# Terminal 2 (start after POX is up)
$ ssh mininet make bench-baseline-mininet
\end{Verbatim}

After the test runs complete, there are two result files that you can
download locally:

\begin{Verbatim}
$ scp mininet:~/pings.txt mininet:~/pings-noflows.txt .
\end{Verbatim}

\section{Monitoring network traffic}
\label{chapter:tcpdump}

If you want to monitor network traffic, you can use the
\texttt{tcpdump}\index{tcpdump}\index{monitoring network traffic}
command.  You specify an interface to listen to with the
\texttt{-i} option (e.g., \textit{S1-eth1};
\texttt{ifconfig}\index{ifconfig} gives a full list) and you can
optionally give packet filtering rules using the
\acf{BPF}-syntax\index{Berkeley Packet Filter}\index{BPF|see{Berkeley
Packet Filter}} \cite{McCanne:1993:BPF:1267303.1267305}.  For instance,

\begin{Verbatim}
$ sudo tcpdump -nNeXS -s64 -iany \
    "(port not 22) and (port not 6633) and (dst 10.0.0.1)"
\end{Verbatim}

specifies in \acs{BPF} to capture packets bound for \texttt{10.0.0.1},
except if the source or destination port is 22
(\textit{ssh}\index{ssh!filtering}) or 6633\index{controller!traffic
monitoring} (the default
controller port).  The option \texttt{-iany} instructs \texttt{tcpdump} to
capture on all interfaces.  The remaining options are explained in the
manual for \texttt{tcpdump}.\footnote{\texttt{man 1 tcpdump}}

Example output of the above command is given below.

\begin{Verbatim}
tcpdump: verbose output suppressed, use -v or -vv for full protocol decode
listening on any, link-type LINUX_SLL (Linux cooked), capture size 64 bytes
20:34:00.063947   P 0e:df:8b:76:a5:1a ethertype IPv4 (0x0800), 
  length 100: 10.0.0.9 > 10.0.0.1: ICMP echo reply, id 15566, seq 868, length 64
  0x0000:  4500 0054 54e4 0000 4001 11bc 0a00 0009  E..TT...@.......
  0x0010:  0a00 0001 0000 0390 3cce 0364 9893 6253  ........<..d..bS
  0x0020:  0000 0000 0284 0000 0000 0000 1011 1213  ................
20:34:00.070384 Out 0e:df:8b:76:a5:1a ethertype IPv4 (0x0800), 
  length 100: 10.0.0.9 > 10.0.0.1: ICMP echo reply, id 15566, seq 868, length 64
  0x0000:  4500 0054 54e4 0000 4001 11bc 0a00 0009  E..TT...@.......
  0x0010:  0a00 0001 0000 0390 3cce 0364 9893 6253  ........<..d..bS
  0x0020:  0000 0000 0284 0000 0000 0000 1011 1213  ................
\end{Verbatim}
