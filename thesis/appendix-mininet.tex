\chapter{Running the thesis code}

You need a virtual machine to run the Mininet VM on.  We have used
VirtualBox.

\section{Mininet}

First you need to download the Mininet VM.
\todo{add link}

The best way to set it up in VirtualBox is to use bridged
networking\todo{more here}.

\section{Setting up ssh}

I have added the following lines to my \texttt{~/.ssh/config}--file:

\begin{verbatim}
Host mininet
  Hostname 192.168.56.102
  User mininet
  ForwardX11 yes
  ForwardAgent yes
\end{verbatim}

This configuration makes sure that we forward our X11 connection and
SSH--agent to mininet.

You should have created an SSH--key and then upload its public part to the
mininet VM:

\begin{verbatim}
$ cat ~/.ssh/id_rsa.pub | ssh mininet cat - >> ~/.ssh/authorized_keys
\end{verbatim}

Now you should be able to log on without any password:

\begin{verbatim}
$ ssh mininet
\end{verbatim}

\section{Cloning the thesis code}

Then do a git clone and install the tools you need. You should already have
been given access to the private github repository by the author.

\begin{Verbatim}
$ ssh mininet
$ git clone https://github.com/cslarsen/bachelor
$ cd bachelor; ./install.sh
\end{Verbatim}

To start the first example \todo{write more}.
