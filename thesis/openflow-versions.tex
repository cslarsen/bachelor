\section{Features in OpenFlow versions 1.0--1.4}

To see how we can enable Paxos functionality in OpenFlow, we need to take a
look at what features it can provide us.

Naturally, we could implement the whole Paxos algorithm in the controller
itself.  Doing so should be quite trivial: One could take an existing
implementation and make some slight modifications to it.  There is in fact
some merit to this idea, because we could deploy systems where most of the
Paxos algorithm is implemented on the controllers, leaving some minor Paxos
code to run on the end--systems (for instance, the end--systems would need
to recognize Paxos LEARN--messages and extract the payload).

However, that would be very inefficient compared to running the entire
algorithm---or parts of it---on the switch.

OpenFlow controllers are meant to set up entries in the flow tables as they
see fit.  In an efficient OpenFlow--system, the
controllers will initially get lots of packets and program the flow tables.
Over time, more work can be performed by the use of flow tables, and so the
controllers will only update the flow tables intermittently.

The flow tables themselves will then be able to match given
packets to the specification and perform actions on them.  These flow tables
are {\em designed} to allow very efficient implementations in hardware.  In
other words, this means that the actions that can be performed are very
simple and far from a fully--fledged programmable system.

Because of this, we need to take a look at what operations we can perform in
OpenFlow.  The versions vary quite a bit on this point, and a lot of
software only support given versions.  So let's get an overview of what is
possible in the different OpenFlow versions.

\section{OpenFlow 1.0}

Prior to OpenFlow 1.0 there were several draft versions not meant for
vendor implementation.  We will not look at these prior versions.  The
following information is taken from \cite{openflow-1.0.0}.

\paragraph{Flow table}
There is only {\em one} flow table.

\paragraph{Counters}
The counters are per--table, per--flow, per--port and per--queue. (TODO: Can
they be used for custom purposes?)

\paragraph{Header fields} 
OpenFlow 1.0 can match on the packet header fields given in table
\ref{table:openflow-1.0.headers}.

\begin{table}
\begin{tabular}{|l|}
\hline \textbf{Header--field for matching} \\
\hline Ingress port \\
\hline Ethernet source address \\
\hline Ethernet destination address \\
\hline VLAN ID \\
\hline VLAN priority \\
\hline IP source address \\
\hline IP destination address \\
\hline IP protocol \\
\hline IP \ac{ToS} bits \\
\hline Transport (TCP or UDP) source port \\
\hline Transport (TCP or UDP) destination port \\
\hline
\end{tabular}
\caption{Header--fields to match on in OpenFlow 1.0}
\label{table:openflow-1.0.headers}
\end{table}

\paragraph{Actions}

\begin{table}
\begin{tabular}{|l|l|l|}
\hline \textbf{Action} &
       \textbf{Required} &
       \textbf{Comment} \\
\hline Forward & Required & All, controller, local, table, in--port \\
\hline Forward & Optional & Normal, flood \\
\hline Enqueue & Optional &  Can be used to implement \ac{QoS} \\
\hline Drop    & Required &  Drops matched packet \\
\hline Modify--field & Optional &  See table \ref{table:openflow-1.0.mods} \\
\hline
\end{tabular}
\caption{Actions in OpenFlow 1.0}
\label{table:openflow-1.0.actions}
\end{table}

\paragraph{Modify--field}

The OpenFlow 1.0 specification strongly recommends that implementors support
the {\em modify--field} action.  The possible modifications are listed in
table \ref{table:openflow-1.0.mods}.

For modifications that invalidate the packet checksums, the switch will
update them.

\begin{table}
\begin{tabular}{|l|l|}
\hline \textbf{Action} \\
\hline Set or replace VLAN ID \\
\hline Set or replace VLAN priority \\
\hline Strip any VLAN header \\
\hline Replace Ethernet source MAC address \\
\hline Replace Ethernet destination MAC address \\
\hline Replace IPv4 source address \\
\hline Replace IPv4 destination address \\
\hline Replace IPv4 \ac{ToS} bits \\
\hline Replace transport (TCP or UDP) source port \\
\hline Replace transport (TCP or UDP) destination port \\
\hline
\end{tabular}
\caption{{\em Modify--field} actions in OpenFlow 1.0}
\label{table:openflow-1.0.mods}
\end{table}
