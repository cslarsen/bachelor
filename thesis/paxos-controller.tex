\section{Paxos in the Controller}

Our first step will be to implement simple Paxos \cite{Lam01} entirely in a
controller\index{controller}.

The aim is to show that a topology with a Paxos-enabled controller will
satisfy the requirements of Paxos---i.e.,~that nodes in the network reach
consensus with progress\todo{Skriv om, og sjekk at forklaring på reqs er
riktig}.

We will use Mininet\index{Mininet} \cite{Lantz:2010:NLR:1868447.1868466} to
run and simulate the \ac{SDN} and POX\index{POX} \cite{POX.1} for
implementing a Paxos in an OpenFlow controller.  POX is part of the
NOX\index{NOX}\index{controller!NOX|see{NOX}}-project \cite{Gude:2008:NTO:1384609.1384625}.  They both use
Python\index{Python}\footnote{Mininet uses Python version 2.7.}
\cite{vanRossum:2009:PRM:1610526} as the implementation language, which
means we can share some code between them.  Both projects are mature and
easy to use.  The Mininet simulation itself will run on a virtual machine
using VirtualBox\index{VirtualBox}.

There are many things to consider when building a working Paxos controller.
We want to provide an internal ordering of packets coming from the \ac{WAN}.
But it would not be wise to do so for \textit{all} packets. For instance, we
need to make sure that \ac{ARP}-packets are working properly, or else
clients will not be able to look up addresses needed for transmission.
If we perform \ac{NAT} and Paxos-ordering on these packets, we quickly need
to perform all kinds of tricks to \acs{ARP}.  In essence, we are dragged
into the minute details of \acs{ARP}.  The same goes for basically any other
protocol that may show up on the network, e.g., \ac{ICMP}.
%
This is clearly out of scope for this
thesis, so we will simply let \acs{ARP} and \acs{ICMP} pass through the
entire network in normal fashion, letting the \acs{L2} switch forward them
to each hop.

Our aim is to mark packets from the \ac{WAN} with the special
\texttt{PAXOS CLIENT} Ethernet type and perform ordering for each one.
It means that, depending on the \ac{MTU}, we would be ordering a lot of
packets.

First we need to be able to route Ethernet frames to their correct
destination. For this we need an \ac{L2} learning switch (see
    \vref{chapter:l2.learning.switch}).

Next, we need to decide which port represents the \ac{WAN}.
We have chosen to create a separate switch and controller for this.
The reason is that it could perform \acf{NAT} so that the Paxos network
would be reachable by a \textit{single} \acs{IP}-address.
We will not build a complete \acs{NAT}, however, but leave it there as a
potential point for implementing such behaviour.

Now, the Paxos-controllers will inspect the packet headers and act only on
those with the correct \texttt{PAXOS} Ethernet type.  If we see a
\texttt{PAXOS CLIENT} message, the leader will initiate an \texttt{ACCEPT}
and, subsequently, \texttt{LEARN}s.

Note that this happens for \textit{every} \ac{TCP} or \ac{UDP} packet.
Depending on the \ac{MTU}, we could potentially perform ordering on a lot of
packets.  This is clearly a downside to the system.  Consider a \acs{TCP}
session. First it will need to establish the \acs{TCP}-connection with a
SYN, then SYN-ACK and a final ACK\index{TCP!three-way handshake}.
All of these will be handled and ordered by the Paxos controllers.
To provide a complete mirroring solution,\footnote{Replication across
end-hosts.} we need to change destination addresses in the Ethernet and
\ac{IP} headers when processing each packet.

There is another important consideration when it comes to switches.
A switch is supposed to be \textit{self-autonomous}, meaning that it should
not require any particular configuration to be able to function. One just
plugs it into the net, and it should start learning which ports each
MAC-address can be reached on.  

This also applies to the Paxos controllers. For them to implement mirroring,
in the sense that each packet coming in from the \ac{WAN} is sent to
each end-host, it needs to modify Ethernet and IP addresses for each packet.
When it starts up, it doesn't know either of these, and has to learn along
the way.  This complicates matters somewhat, and using a configuration file
with a full map of the network would be contrary to the principle of
self-autonomy.  What we have done is to let the L2 switch track both
Ethernet and IP-addresses. When each Paxos node announces itself, we learn
which ports the Paxos nodes are on. This is important for the
WAN-controller, who needs to shuttle packets between each network.

Each Paxos controller will then know which port the WAN is on (only applies
for leader) and which ports other Paxos controllers are on. It can therefore
deduce that the remaining ports are links to their end-hosts.

Messages going from the client to the end-hosts are ordered by the Paxos
system. But packets going the other way, from end-hosts to the clients, pass
through the network unhindered.  The WAN-controller should be able to pick
only one of these.\todo{Finn ut hvordan vi gjør dette.}

\todo{Skriv mere!}
\todo{Få med at vi må liksom forholde oss til veldig mange protokoller,
  som tcp udp osv, og arp, for det er så mye som foregår}

