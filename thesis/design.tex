\chapter{Design}
\label{chapter:design}

Now that we have discussed the main algorithms and our simplification of
them, we must take a look at what OpenFlow can offer us to reach our goals.

\section{Features in OpenFlow versions 1.0--1.4}

To see how we can enable Paxos functionality in OpenFlow, we need to take a
look at what features it can provide us.

Naturally, we could implement the whole Paxos algorithm in the controller
itself.  Doing so should be quite trivial: One could take an existing
implementation and make some slight modifications to it.  There is in fact
some merit to this idea, because we could deploy systems where most of the
Paxos algorithm is implemented on the controllers, leaving some minor Paxos
code to run on the end--systems (for instance, the end--systems would need
to recognize Paxos LEARN--messages and extract the payload).

However, that would be very inefficient compared to running the entire
algorithm---or parts of it---on the switch.

OpenFlow controllers are meant to set up entries in the flow tables as they
see fit.  In an efficient OpenFlow--system, the
controllers will initially get lots of packets and program the flow tables.
Over time, more work can be performed by the use of flow tables, and so the
controllers will only update the flow tables intermittently.

The flow tables themselves will then be able to match given
packets to the specification and perform actions on them.  These flow tables
are {\em designed} to allow very efficient implementations in hardware.  In
other words, this means that the actions that can be performed are very
simple and far from a fully--fledged programmable system.

Because of this, we need to take a look at what operations we can perform in
OpenFlow.  The versions vary quite a bit on this point, and a lot of
software only support given versions.  So let's get an overview of what is
possible in the different OpenFlow versions.




\section{Limitations in OpenFlow}

By looking at what OpeFlow versions 1.1--1.3 offer, one can see that we can't really
make use of any of the added functionality for running Paxos.  What we need
is the ability to run programs on the switch, which is something OpenFlow
does not support at all.  Neither do their action primitives add up to
anything that could be used for remembering state (such as the current round
number) or executing if--then--else statements.

One possibly solution would be to insert a lot of flow table entries that
each waited for a specific round number.  But that would not be an elegant
or practical solution.

They \textit{do}, however, offer us the ability to implement Paxos entirely
on the controller.  We have actually done this, but one of the stated goals
of this thesis was to move parts of the Paxos code down to the switch
itself.  For this we simply need to be able to run full Turing--equivalent
programming languages.

For remembering state, we looked at the metadata that is available in later
versions of OpenFlow.  However, metadata only exists as the packet is
processed in the pipeline of flow tables, and is erased when the packet
actions are applied at the end.  To remember state, we will need to add a
table to hold such data in the switch.

Finally, we have to look at which OpenFlow versions our software components
support.

Mininet seems to support whatever version of OpenFlow that OpenVSwitch uses,
as this is what it uses as a switch.  OpenVSwitch It supports OpenFlow versions
1.0---1.3 almost fully, but support for 1.4 is flaky, and may crash.  So 1.4
is out of the question.

The most obvious component to look at is POX, our controller
framework in Python, which only supports OpenFlow 1.0\footnote{It does seem
to support some \textit{Nicira extensions}, though.  These are extensions
that were originally added to early OpenFlow versions, but much of it
has been implemented in later versions.  There is also a fork of POX (and other
software projects) written by CPqD that adds support for newer OpenFlow
versions, but we haven't looked at it.}.

But the major point for our decision is what OpenFlow can offer us.
There simply is no way of executing general code, and there is no way to
remember state\footnote{We even investigated whether we could use the
counters to count round numbers or store them in IPv6 addresses, using VLAN
for storing data, etc.  All those ideas turned out to be very hairy to
implement, with a real possibility of not working correctly.}

All in all, we have decided to use OpenFlow 1.0 where applicable and extend
it where needed.  Using the flow table, controller and bytecode is a simple
but good, practical decision.

\section{Decisions}

We have seen that OpenFlow does not offer the capabilities we need to
implement Paxos on the switch.  We could implement Paxos on the controller,
but that would be nearly equivalent to having adding Paxos to the software
running on the end--hosts.

Thus, we have decided to use a combination of OpenFlow flow table rules and
programs running in OpenVSwitch as bytecode compiled fragments to handle the
details of the Paxos algorithms.

\todo{Legg til selve designet her, siden tittelen sier det! Vis hvor vi
bruker flow tables (ikke vis detaljer, det kommer senere), vis
nettverksflyt, vis hvor bytecode blir eksekvert, og hva som kjører på
controller.}

\section{Choice of switch programming language}

\todo{Insert a "defense" of why we chose Forth, and why we didn't implement
everything in OpenFlow}

What we want is to provide \textit{simple} primitives that can be
implemented to run custom code \textit{efficiently} on the
hardware---requiring little memory and few cycles per operation---while
still being useful for other networking protocols.
\todo{back up this statement on the last part}

We think this is a good implementation for fastidious hardware implementors,
but \textit{any} programming language---preferably one that can produce
bytecode---would work just as fine.
\todo{Finn ut hvor jeg snakker om Ngaro og flytt tekst over her.}

